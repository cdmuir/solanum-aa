% Options for packages loaded elsewhere
\PassOptionsToPackage{unicode}{hyperref}
\PassOptionsToPackage{hyphens}{url}
\PassOptionsToPackage{dvipsnames,svgnames,x11names}{xcolor}
%
\documentclass[
  letterpaper,
  DIV=11,
  numbers=noendperiod]{scrartcl}

\usepackage{amsmath,amssymb}
\usepackage{iftex}
\ifPDFTeX
  \usepackage[T1]{fontenc}
  \usepackage[utf8]{inputenc}
  \usepackage{textcomp} % provide euro and other symbols
\else % if luatex or xetex
  \usepackage{unicode-math}
  \defaultfontfeatures{Scale=MatchLowercase}
  \defaultfontfeatures[\rmfamily]{Ligatures=TeX,Scale=1}
\fi
\usepackage{lmodern}
\ifPDFTeX\else  
    % xetex/luatex font selection
  \setmainfont[]{Times New Roman}
\fi
% Use upquote if available, for straight quotes in verbatim environments
\IfFileExists{upquote.sty}{\usepackage{upquote}}{}
\IfFileExists{microtype.sty}{% use microtype if available
  \usepackage[]{microtype}
  \UseMicrotypeSet[protrusion]{basicmath} % disable protrusion for tt fonts
}{}
\makeatletter
\@ifundefined{KOMAClassName}{% if non-KOMA class
  \IfFileExists{parskip.sty}{%
    \usepackage{parskip}
  }{% else
    \setlength{\parindent}{0pt}
    \setlength{\parskip}{6pt plus 2pt minus 1pt}}
}{% if KOMA class
  \KOMAoptions{parskip=half}}
\makeatother
\usepackage{xcolor}
\setlength{\emergencystretch}{3em} % prevent overfull lines
\setcounter{secnumdepth}{-\maxdimen} % remove section numbering
% Make \paragraph and \subparagraph free-standing
\ifx\paragraph\undefined\else
  \let\oldparagraph\paragraph
  \renewcommand{\paragraph}[1]{\oldparagraph{#1}\mbox{}}
\fi
\ifx\subparagraph\undefined\else
  \let\oldsubparagraph\subparagraph
  \renewcommand{\subparagraph}[1]{\oldsubparagraph{#1}\mbox{}}
\fi


\providecommand{\tightlist}{%
  \setlength{\itemsep}{0pt}\setlength{\parskip}{0pt}}\usepackage{longtable,booktabs,array}
\usepackage{calc} % for calculating minipage widths
% Correct order of tables after \paragraph or \subparagraph
\usepackage{etoolbox}
\makeatletter
\patchcmd\longtable{\par}{\if@noskipsec\mbox{}\fi\par}{}{}
\makeatother
% Allow footnotes in longtable head/foot
\IfFileExists{footnotehyper.sty}{\usepackage{footnotehyper}}{\usepackage{footnote}}
\makesavenoteenv{longtable}
\usepackage{graphicx}
\makeatletter
\def\maxwidth{\ifdim\Gin@nat@width>\linewidth\linewidth\else\Gin@nat@width\fi}
\def\maxheight{\ifdim\Gin@nat@height>\textheight\textheight\else\Gin@nat@height\fi}
\makeatother
% Scale images if necessary, so that they will not overflow the page
% margins by default, and it is still possible to overwrite the defaults
% using explicit options in \includegraphics[width, height, ...]{}
\setkeys{Gin}{width=\maxwidth,height=\maxheight,keepaspectratio}
% Set default figure placement to htbp
\makeatletter
\def\fps@figure{htbp}
\makeatother
% definitions for citeproc citations
\NewDocumentCommand\citeproctext{}{}
\NewDocumentCommand\citeproc{mm}{%
  \begingroup\def\citeproctext{#2}\cite{#1}\endgroup}
\makeatletter
 % allow citations to break across lines
 \let\@cite@ofmt\@firstofone
 % avoid brackets around text for \cite:
 \def\@biblabel#1{}
 \def\@cite#1#2{{#1\if@tempswa , #2\fi}}
\makeatother
\newlength{\cslhangindent}
\setlength{\cslhangindent}{1.5em}
\newlength{\csllabelwidth}
\setlength{\csllabelwidth}{3em}
\newenvironment{CSLReferences}[2] % #1 hanging-indent, #2 entry-spacing
 {\begin{list}{}{%
  \setlength{\itemindent}{0pt}
  \setlength{\leftmargin}{0pt}
  \setlength{\parsep}{0pt}
  % turn on hanging indent if param 1 is 1
  \ifodd #1
   \setlength{\leftmargin}{\cslhangindent}
   \setlength{\itemindent}{-1\cslhangindent}
  \fi
  % set entry spacing
  \setlength{\itemsep}{#2\baselineskip}}}
 {\end{list}}
\usepackage{calc}
\newcommand{\CSLBlock}[1]{\hfill\break\parbox[t]{\linewidth}{\strut\ignorespaces#1\strut}}
\newcommand{\CSLLeftMargin}[1]{\parbox[t]{\csllabelwidth}{\strut#1\strut}}
\newcommand{\CSLRightInline}[1]{\parbox[t]{\linewidth - \csllabelwidth}{\strut#1\strut}}
\newcommand{\CSLIndent}[1]{\hspace{\cslhangindent}#1}

\usepackage{booktabs}
\usepackage{longtable}
\usepackage{array}
\usepackage{multirow}
\usepackage{wrapfig}
\usepackage{float}
\usepackage{colortbl}
\usepackage{pdflscape}
\usepackage{tabu}
\usepackage{threeparttable}
\usepackage{threeparttablex}
\usepackage[normalem]{ulem}
\usepackage{makecell}
\usepackage{xcolor}
\KOMAoption{captions}{tableheading}
\usepackage{hyperref}
\addtokomafont{disposition}{\rmfamily}
\usepackage{siunitx}
\newcommand{\aax}{$\mathrm{AA}$}
\newcommand{\Aamphi}{$A_{\mathrm{amphi}}$}
\newcommand{\Ahypo}{$A_{\mathrm{hypo}}$}
\newcommand{\agcurve}{$A \textendash g_\text{sw}$}
\newcommand{\aqcurve}{$A \textendash Q$}
\newcommand{\ca}{$C_\mathrm{a}$}
\newcommand{\caequals}[1]{$C_\mathrm{a} = \qty{#1}{\micro\mol\raiseto{-1}\mol}$}
\newcommand{\cabetween}[2]{#1 < $C_\mathrm{a} < \qty{#2}{\micro\mol\raiseto{-1}\mol}$}
\newcommand{\gsw}{$g_\text{sw}$}
\newcommand{\loggsw}{$\log(g_\text{sw})$}
\newcommand{\logA}{$\log(A)$}
\newcommand{\ppfd}{$\mathrm{PPFD}$}
\newcommand{\ppfdequals}[1]{$\mathrm{PPFD} = \qty{#1}{\micro\mol\raiseto{-2}\meter\raiseto{-1}\second}$}
\newcommand{\ppfdqty}[1]{$\qty{#1}{\micro\mol\raiseto{-2}\meter\raiseto{-1}\second}$}
\newcommand{\rh}{$\mathrm{RH}$}
\newcommand{\rhequals}[1]{$\mathrm{RH} = #1\%$}
\newcommand{\tleafequals}[1]{$T_\mathrm{leaf} = \qty{#1}{\degreeCelsius}$}
\makeatletter
\@ifpackageloaded{caption}{}{\usepackage{caption}}
\AtBeginDocument{%
\ifdefined\contentsname
  \renewcommand*\contentsname{Table of contents}
\else
  \newcommand\contentsname{Table of contents}
\fi
\ifdefined\listfigurename
  \renewcommand*\listfigurename{List of Figures}
\else
  \newcommand\listfigurename{List of Figures}
\fi
\ifdefined\listtablename
  \renewcommand*\listtablename{List of Tables}
\else
  \newcommand\listtablename{List of Tables}
\fi
\ifdefined\figurename
  \renewcommand*\figurename{Figure}
\else
  \newcommand\figurename{Figure}
\fi
\ifdefined\tablename
  \renewcommand*\tablename{Table}
\else
  \newcommand\tablename{Table}
\fi
}
\@ifpackageloaded{float}{}{\usepackage{float}}
\floatstyle{ruled}
\@ifundefined{c@chapter}{\newfloat{codelisting}{h}{lop}}{\newfloat{codelisting}{h}{lop}[chapter]}
\floatname{codelisting}{Listing}
\newcommand*\listoflistings{\listof{codelisting}{List of Listings}}
\makeatother
\makeatletter
\makeatother
\makeatletter
\@ifpackageloaded{caption}{}{\usepackage{caption}}
\@ifpackageloaded{subcaption}{}{\usepackage{subcaption}}
\makeatother
\ifLuaTeX
  \usepackage{selnolig}  % disable illegal ligatures
\fi
\usepackage{bookmark}

\IfFileExists{xurl.sty}{\usepackage{xurl}}{} % add URL line breaks if available
\urlstyle{same} % disable monospaced font for URLs
\hypersetup{
  pdftitle={Unknown},
  colorlinks=true,
  linkcolor={blue},
  filecolor={Maroon},
  citecolor={Blue},
  urlcolor={Blue},
  pdfcreator={LaTeX via pandoc}}

\title{Unknown}
\author{}
\date{}

\begin{document}
\maketitle

\begin{verbatim}

Attaching package: 'dplyr'
\end{verbatim}

\begin{verbatim}
The following objects are masked from 'package:stats':

    filter, lag
\end{verbatim}

\begin{verbatim}
The following objects are masked from 'package:base':

    intersect, setdiff, setequal, union
\end{verbatim}

\subsection{Notes}\label{notes}

In preamble: - short codes for commonly used symbols - ?

Terminology to standardize:

\begin{itemize}
\item
  the native light habitat or environment: the SPLASH paper refers to
  what they calculated as `habitat' PPFD, so I am going to use that
  terminology
\item
  growth condition versus measurement conditions
\item
  ?
\end{itemize}

\subsection{Authors}\label{authors}

definite: Wei Shen Lim, Dachuan Wang, Chris Muir question: Sam McKlin,
Tom Buckley

\subsection{Main}\label{main}

{[}introduction{]} Stomata are important

Stomatal ratio is important unsolved problem

Amphi leaves are more common in sunny habitats and confer a great
benefit. (this pattern is also true in Solanum? could we analyze
herbarium data here?)

{[}maybe make distinction between amphi leaves and var in SR here since
all tomato species are amphi{]}

Leaves with greater stomatal density ratio are more in open, sunny
habitats because they deliver the greatest benefit in those
circumstances. An amphistomatous leaf increases photosynthetic carbon
gain compared to an otherwise identical hypostomatous leaf by increasing
conductance through the leaf intercellular airspaces and boudary layers;
the additional water loss through a second boundary layer is typically
small {[}cites{]}. We quantify this benefit as the amphistomy advantage
(AA = log(Aamphi/Ahypo). Why would AA be greater in sun than shade?
There are three nonmutually exclusive hypotheses that we classify as
`acclimatory', `plastic', and `constitutive'.

Acclimatory hypothesis: Leaves acclimated to high light intensity
typically increase total leaf stomatal conductance (increased CO\$\_2\$
supply) and upregulate Rubisco activation (increased CO\$\emph{2\$}
demand). A one-dimensional circuit model using the Farquhar-von
Caemmerer-Berry biochemical model of C\$\_3\$ photosynthesis shows that
both increased stomatal conductance and Rubisco activity should increase
AA, all else being equal (Supporting Information). If the acclimatory
hypothesis is correct, we predict that AAhigh \textgreater{} AAlow for
all species regardless of native habitat or growth environment. Plants
adapted to sunny, open habitats will evolve greater stomatal density
ratio to take advantage of regular exposure to high light intensity.

Plastic hypothesis: Individuals of the same genotype often develop
dramatically different leaves in sun and shade conditions (cite).
Plastic responses are likely adaptations to optimize photosynthesis at
different light intensities (cites), but changes in leaf anatomy and
biochemistry could modulate AA as a byproduct. Thicker or less porous
leaves, both of which are associated with high leaf mass per area (LMA),
will have lower g\_ias; leaves with increased total stomatal density and
Rubisco concentration have greater potential CO2 supply and demand.
Under the plastic hypothesis, we predict that AAsun \textgreater{}
AAshade for all species and light intensities. AAsun and gsmax,sun
should be positively associated with native light habitat.
{[}transition{]} We assume that genotypes adapted to sunny, open
habitats will express a phenotype best adapted to that environment when
leaves develop under high light intensity; genotypes adapted to shaded
closed habitats may be plastic, but suboptimal for light intensities
they do not regularly experience in nature.

{[}conceptual figure could also show differential benefit of amphi
leaves in sun/shade{]}

to plants growing there. in those It has been hypothesized that
amphistomy increases photosynthesis more in sunny places. Three
nonmutually exclusive adaptive explanations for why amphi leaves:

Conceptual figure explaining each hypothesis

\begin{itemize}
\tightlist
\item
  acclimation: greater demand, higher gs increase AA
\item
  developmental plasticity: light-induced changes in leaf anatomy
  modulate AA
\item
  constitutive: genetic differences in leaves adapted to different light
  habitats
\end{itemize}

We distinguished among these hypotheses by comparing AA among wild
tomato species from different native light habitats, grown under
simulated sun and shade light treatments, and measured under contrasting
light intensity (Figure of hypotheses and predictions). We measured AA
on 600 individual plants from 30 accessions (average of 10 replicates
per light treatment) using a recently developed method (\emph{1}). With
this method, we directly compare the photosynthetic rate of an untreated
amphistomatous leaf to that of the same leaf with gas exchange blocked
through the adaxial (upper) surface by transparent plastic, which we
refer to as `pseudohypostomy'. To compare amphi- and pseudohypostomatous
leaves at identical whole-leaf \(g_\text{sc}\), we measure \(A\) over a
range of \(g_\text{sc}\), inducing stomatal opening and closure by
modulating humidity (see Materials and Methods for further details).

Table of directional predictions (with table summarizing results? part
of conceptual figure? in supplement?)

caption: Directional predictions associated with each hypothesis
explaining why amphistomy advantage (AA) might be greater for leaves in
sunny, open habitats. For each hypothesis, we make predictions for how
native light habitat, light treatment, and light intensity would affect
AA.

\begin{longtable}[]{@{}
  >{\centering\arraybackslash}p{(\columnwidth - 6\tabcolsep) * \real{0.1972}}
  >{\centering\arraybackslash}p{(\columnwidth - 6\tabcolsep) * \real{0.3099}}
  >{\centering\arraybackslash}p{(\columnwidth - 6\tabcolsep) * \real{0.2535}}
  >{\centering\arraybackslash}p{(\columnwidth - 6\tabcolsep) * \real{0.2394}}@{}}
\toprule\noalign{}
\begin{minipage}[b]{\linewidth}\centering
hypothesis
\end{minipage} & \begin{minipage}[b]{\linewidth}\centering
native light habitat
\end{minipage} & \begin{minipage}[b]{\linewidth}\centering
light treatment
\end{minipage} & \begin{minipage}[b]{\linewidth}\centering
light intensity
\end{minipage} \\
\midrule\noalign{}
\endhead
\bottomrule\noalign{}
\endlastfoot
acclimatory & cor(PAR, AA) = 0 & AAsun = AAshade & AA2000 \textgreater{}
AA150 \\
plastic & cor(PAR, AA) = 0 & AAsun \textgreater{} AAshade & AA2000 =
AA150 \\
constitutive & cor(PAR, AA) \textgreater{} 0 & AAsun = AAshade & AA2000
= AA150 \\
\end{longtable}

{[}add rows for when multiple hypotheses are supported simultaneously?
or put that in SI?{]}

{[}results{]} Amphistomy increases \(A\) in all accessions, in both sun
and shade leaves, and light intensities. We infer this from the fact
that blocking gas exchange in pseudohypostomatous leaves reduced \(A\)
by X-X\% depending on the accession, light treatment, and light
intensity (Table/figure). The AA is equivalent to an X-X\% change in
total \(g_\text{sc}\) (see SI section gs equivalency). But whereas
increasing \(g_\text{sc}\) would increase water loss as a necessary
by-product, amphistomy can increase \(A\) without any appreciable affect
on transpiration.

Sun leaves from high light habitats {[}not sure if htis result is true
yet{]} benefit the most from amphistomy because of a both developmental
plasticity and constitutive differences among accessions. {[}quantify
difference in AA and contribution of different affects{]}. Surprisingly,
light intensity had a little effect\ldots{} (should this be in this
paragraph, or it's own?).

{[}discussion{]} - sun/shade has long been appreciated and this shows
new trait that should be considered and that CO2 diffusion becomes major
limitation

\subsection{Materials and Methods}\label{materials-and-methods}

{[}this will be moved to SI eventually{]}

\subsubsection{Accessions}\label{accessions}

We compared AA among 30 ecologically diverse accessions of wild tomato,
including representatives of all described species of \emph{Solanum}
sect. \emph{Lycopersicon} and sect. \emph{Lycopersicoides} (\emph{2})
and the cultivated tomato \emph{S. lycopersicum} var.
\emph{lycopersicum} \autoref{tab:accessions}. Due to constraints on
growth space and time, we spread out measurements over 80 weeks from May
1, 2022 to October 31, 2023. Replicates within accession were evenly
spread out over this period to prevent confounding of temporal variation
in growth conditions with accession. {[}anything else to say here? maybe
explain accession selection and phylogeny?{]}

\begin{table}

\caption{\label{tbl-accessions}Solanum accessions}

\centering{

}

\end{table}%

\subsubsection{Plant growth conditions}\label{plant-growth-conditions}

In all growth spaces, we recorded \(\mathrm{PPFD}\) using full spectrum
quantum sensors (SQ-500-SS, Apogee Instruments, Logan, Utah, USA); we
recorded temperature, RH, and {[}CO\(_2\){]} using an EE894 sensor (E+E
Elektronik, Engerwitzdorf, Austria) protected by a radiation shield. All
environmental measurements were taken every 10 minutes from the middle
of plants racks at approximately the same height as the leaves we
measured. We measured leaf temperature of focal leaves prior to
measurement using an infrared radiometer (SI-111-SS, Apogee Instruments,
Logan, Utah, USA).

\paragraph{Germination and seedling
stage}\label{germination-and-seedling-stage}

Seeds provided by the Tomato Genetics Resource Center germinated on
moist paper in plastic boxes after soaking for 30-60 minutes in a 50\%
(volume per volume) solution of household bleach and water, followed by
a thorough rinse. We transferred seedlings to cell-pack flats containing
Pro-Mix BX potting mix (Premier Tech, Rivière-du-Loup, Quebec, Canada)
once cotyledons fully emerged, typically within 1-2 weeks of sowing. We
grew seeds and seedlings for both sun and shade treatments under the
same environmental conditions (12:12 h, 25:\(\qty{20}{\degreeCelsius}\),
40:60 RH day:night cycle). LED light provided
\(\mathrm{PPFD} = \qty{200}{\micro\mol\raiseto{-2}\meter\raiseto{-1}\second}\)
(Fluence RAZRx, Austin, Texas, USA).

\paragraph{Light treatments}\label{light-treatments}

Seedlings were randomly assigned in alternating order within accession
to the sun or shade treatment during transplanting. After seedlings
established in cell-pack flats for \(\approx 2\) weeks, we transplanted
them to 3.78 L plastic pots containing 60\% Pro-Mix BX potting mix, 20\%
coral sand (Pro-Pak, Honolulu, Hawaiʻi, USA), and 20\% cinders (Niu
Nursery, Honolulu, Hawaiʻi, USA). Percentage composition is on a volume
basis. The soil mixture contained slow release NPK fertilizer following
manufacturer instructions (Osmocote Smart-Release Plant Food Flower \&
Vegetable, The Scotts Company, Marysville, Ohio, USA). We determined pot
field capacity one week after transplanting using a scale (Ohaus V12P15
Valor 1000, Parsippany, New Jersey, USA) and watered to field capacity
three times per week to prevent drought stress.

We assigned sun and shade treatment to lower and upper racks of a
\(\qty{1.22}{\meter} \times \qty{2.44}{\meter}\) shelving unit in a
climate-controlled growth room. We assigned the sun treatment to the
lower rack to limit diffuse light from reaching the shade treatment. The
average daytime \ppfd{} was \ppfdqty{800} and \ppfdqty{125} for sun and
shade treatments, respectively. To isolate the effect of light intensity
from quality, we used the same LED model with the the same spectrum
(Fluence SPYDR 2i, Austin, Texas, USAS), but dimmed the lights in the
shade treatment. To maintain homogeneous environmental conditions other
than light, we mixed air within the growth room using an air circulator
(Vornado 693DC, Andover, Kansas, USA) and within racks using a miniature
oscillating air circulator (Vornado Atom 1, Andover, Kansas, USA).
Despite these efforts, the air in the sun treatment was on average
\(\qty{1}{\degreeCelsius}\) warmer and the average RH was consequently 5
lower. However, because of evaporative cooling, the leaves in the sun
treatment were only \(\qty{1}{\degreeCelsius}\) on average (\(n = 600\)
leaves).

\subsubsection{Leaf trait measurements}\label{leaf-trait-measurements}

We selected a fully expanded, unshaded leaf at least six leaves above
the cotyledons during early vegetative growth. This typically meant that
plants had grown in light treatments for \(\approx\) 4 weeks, ensuring
they had time to sense and respond developmentally to the light
intensity of the treatment rather than the seedling conditions
(\emph{3}). Shade plants grew slower than sun plants, hence leaves at
the same developmental stage were measured on chronologically older
plants in the shade treatment. In some sun plants, we had to use leaves
higher on the stem because short internodes made lower leaves
inaccessible with the gas exchange equipment. We measured terminal
leaflets in 70.0\% of cases, but used the lateral leaflet closest to the
terminal leaflet when it was damaged or difficult to clamp into the gas
exchange chamber. When a leaflet was damaged during gas exchange
measurements, we collected anatomical data from the nearest leaflet on
the same leaf (10.0)\%. We measured chlorophyll concentration index
(CCI) using a chrolophyll concentration meter (MC-100, Apogee
Instruments, Logan, Utah, USA) on the lamina of focal leaflets before
gas exchange measurements at the same time we measured leaf temperature.

\paragraph{Amphistomy advantage}\label{amphistomy-advantage}

We estimated `amphistomy advantage' (\aax) \emph{sensu} (\emph{4}), but
with modifications previously described in (\emph{1}). \aax{} is
calculated as the log-response ratio of \(A\) compared at the same total
\gsw:

\[\mathrm{AA} = \mathrm{log}(A_{\mathrm{amphi}} / A_{\mathrm{hypo}})\]

We measured the photosynthetic rate of an untreated amphistomatous leaf
(\Aamphi) over a range of \gsw{} values. We refer to this as an
\agcurve{} curve. We compared the \agcurve{} curve of the untreated leaf
to the photosynthetic rate of pseudohypostomatous leaf (\Ahypo), which
is the same leaf but with gas exchange through the upper surface blocked
by a neutral density plastic (propafilm).

We measured \agcurve{} curves using a portable infrared gas analyzer
(LI-6800PF, LI-COR Biosciences, Lincoln, Nebraska, USA).
Light-acclimated plants were placed under LEDs dimmed to match their
light treatment during gas exchange measurements. We estimated the
photosynthetic rate (\(A\)) and stomatal conductance to CO\(_2\) (\gsw)
at ambient CO\(_2\) (\caequals{415}) and \tleafequals{25.0}. The
irradiance of the light source in the pseudohypo leaf was higher because
the propafilm reduces transmission. To compensate for reduced
transmission, we increased incident \ppfd for pseudohypo leaves by a
factor 1/0.91, the inverse of the measured transmissivity of the
propafilm. We also set the stomatal conductance ratio, for purposes of
calculating boundary layer conductance, to 0 for pseudohypo leaves
following manufacturer directions.

We collected four \agcurve{} curves per leaf, an amphi (untreated) curve
and a pseudohypo (treated) curve at high light-intensity
(\ppfdequals{2000}; 97.7:2.3 red:blue) and low light-intensity
(\ppfdequals{150}; 87.0:13.0 red:blue). We always measured high
light-intensity curves first because photosynthetic downregulation is
faster than upregulation in these species. To control for order effects,
we alternated between starting with amphi or pseudohypo leaf
measurements. Unlike (\emph{1}), preliminary experiments with
\emph{Solanum} indicated a strong order effect in that \(A\) declined in
the second curve. Therefore, we made measurements over two days. On the
first day, we measured high and low light-intensity curves for either
amphi or pseudohypo leaves; on the second day, we measured high and low
light-intensity curves on the other leaf type.

In all cases, we acclimated the focal leaf to high light
(\ppfdequals{2000}) and high relative humidity (\rhequals{70}) until
\(A\) and \gsw{} reach their maximum. After that, we decreased \rh{} to
\(\approx 10\%\) to induce rapid stomatal closure without biochemical
downregulation. Hence, \Aamphi{} and \Ahypo{} were both measured at low
chamber humidity after the leaf had acclimated to high humidity. All
other environmental conditions in the leaf chamber remained the same. We
logged data until \gsw{} reached its nadir. We then acclimated the leaf
to low light (\ppfdequals{150}) and \rhequals{70} before inducing
stomatal closure with low \rh and logging data as described above.

\paragraph{\texorpdfstring{Light-response (\aqcurve)
curves}{Light-response () curves}}\label{light-response-curves}

In 90.0\% of plants, we measured light-response (\aqcurve) curves on the
same leaflets as \agcurve{} curves. However, when a leaflet was damaged
during \agcurve{} curves, we used the next closest leaflet for
\aqcurve{} curves. Leaves acclimated to high light-intensity
(\ppfdequals{2000}), ambient CO\(_2\) (\caequals{415}), \rhequals{50},
and \tleafequals{25}. After \(A\) and \gsw{} stabilized, we measured
\(A\) at 20 light-intensity levels between \(0\) and
\(\qty{2000}{\micro\mol\raiseto{-2}\meter\raiseto{-1}\second}\) in
descending order.

\paragraph{Stomatal anatomy}\label{stomatal-anatomy}

We estimated the stomatal density and size on ab- and adaxial leaf
surfaces from all leaves, using guard cell length as a proxy for
stomatal size since it proportional to maximum conductance (\emph{5}).
We made surface impressions of leaf lamina from the same area used for
gas exchange measurements using a-silicone impression material (Zhermack
elite HD+, light body, fast set, Rovigo, Italy). We applied clear nail
polish to make positive replicas of the impression. After nail polish
dried, we mounted replicas on a microscope slide using transparent tape
(\emph{6}). We digitized a portion of each leaf surface replica using a
brightfield microscope (Leica DM2000, Wetzlar, Germany). We counted and
measured guard cell length on all stomata using the FIJI implementation
of ImageJ2 version 2.3.0 (\emph{7}), then divided the count by the
visible leaf area (\(\qty{0.890}{\raiseto{2}\milli\meter}\)) to estimate
stomatal density.

\paragraph{Leaf mass per area}\label{leaf-mass-per-area}

Leaf mass per area (LMA) is the dry mass divided by the leaflet area. We
scanned fresh leaflets on a flat bed scanner (Epson V600, Los Alamitos,
California. USA) and measured leaflet area from digital images using the
FIJI implementation of ImageJ2 version 2.3.0 (\emph{7}). We dried leaves
for 72 hours at \(\qty{74}{\degreeCelsius}\) in a food dehydrator
(Cosori CP267-FD, Vesync Co., Anaheim, California, USA) and weighed
using a benchtop analytical balance (Ohaus PR64 Analytical Balance,
Parsippany, New Jersey, USA). In \(5.0\%\) we measured LMA on the
adjacent leaflet because the focal leaflet was damaged or wilted while
making surface impressions and we could not reliably estimate area. LMA
data are missing from \(3.0\%\) of individuals because the area or mass
was not recorded at all or recorded incorrectly.

\subsubsection{\texorpdfstring{Cleaning \agcurve{}
curves}{Cleaning  curves}}\label{cleaning-curves}

The raw data set consisted of 2,370 \agcurve{} curves with an average of
63.2 points per curve. Manual curation of a data set this size in a
principled, consistent manner is not feasible. Therefore, we automated
data cleaning using custom \emph{R} scripts. Cleaning is divided into
six sequential steps \autoref{tbl-cleaning1}.

\begin{table}

\caption{\label{tbl-cleaning1}Six sequential steps for cleaning
\agcurve{} curves. The rationale and procedure for each step are
described in the text. The rightmost columns summarize the number of
curves and mean number of points per curve remaining after each step.
For reference, there are four possible \agcurve{} curves per replicate:
all combinations of leaf type (amphi or pseudohypo) and light intensity
(high or low).}

\centering{

\centering
\begin{tabular}{>{\raggedright\arraybackslash}p{9cm}>{\centering\arraybackslash}p{2.5cm}>{\centering\arraybackslash}p{2.5cm}}
\toprule
Step: description & Number of curves & Number of points per curve\\
\midrule
\cellcolor{gray!10}{1. remove unreliable and unusable data points} & \cellcolor{gray!10}{2,361} & \cellcolor{gray!10}{63.0}\\
2. remove hysteretic portion of \agcurve{} curves at low \gsw{} & 2,360 & 59.2\\
\cellcolor{gray!10}{3. remove outliers within each \agcurve{} curve} & \cellcolor{gray!10}{2,360} & \cellcolor{gray!10}{58.7}\\
4. remove replicates with no overlap between amphi and pseudohypo \agcurve{} curves & 2,268 & 58.5\\
\cellcolor{gray!10}{5. thin redundant data points within each \agcurve{} curve} & \cellcolor{gray!10}{2,268} & \cellcolor{gray!10}{28.1}\\
\addlinespace
6. Trim extreme \aax{} values & 2,210 & 28.0\\
\bottomrule
\end{tabular}

}

\end{table}%

\paragraph{Remove unreliable and unusable data
points}\label{remove-unreliable-and-unusable-data-points}

\emph{Rationale}: Unreliable data points consisted of those where
chamber {[}CO\(_2\){]} was unstable and therefore measurements are not
biologically meaningful. Unusable data points were those where \(A < 0\)
because the logarithm of a negative number is undefined.

\emph{Procedure}: We retained data points where \cabetween{410}{420} and
\(A> 0\).

\paragraph{\texorpdfstring{Remove hysteretic portion of \agcurve{}
curves at low
\gsw{}}{Remove hysteretic portion of  curves at low }}\label{remove-hysteretic-portion-of-curves-at-low}

\emph{Rationale}: In most \agcurve{} curves, we observed a hysteretic
response at low \gsw. After \gsw{} and \(A\) declined simultaneously,
\(A\) increased slightly as \gsw{} continued to decline or stabilize,
indicating some leaf acclimation to low \rh. We removed this portion of
the curve to focus curve-fitting on the primary domain where \(A\)
increases monotonically with \gsw{}.

\emph{Procedure}: For each curve, we removed data points after \gsw{}
had reached its minimum unless there were fewer than 10 data points
remaining.

\paragraph{\texorpdfstring{Remove outliers within each \agcurve{}
curve}{Remove outliers within each  curve}}\label{remove-outliers-within-each-curve}

\emph{Rationale}: Individual outliers within \agcurve{} curves, usually
caused by transitory changes in chamber conditions, exert undue leverage
on parameter estimates and cause bias and/or low precision in parameter
estimates.

\emph{Procedure}: We fit provisional quadratic regressions for each
curve using ordinary least squares with the \texttt{lm()} function in
\emph{R}. We sequentially removed data points with an absolute external
studentized residual \(> 3\) until none remained.

\paragraph{\texorpdfstring{Thin redundant data points within each
\agcurve{}
curve}{Thin redundant data points within each  curve}}\label{thin-redundant-data-points-within-each-curve}

\emph{Rationale}: Data points closely spaced along the \agcurve{} curve
provide redundant information and may be highly correlated
(i.e.~pseudoreplication). This occurred because data was logged at a
constant temporal interval, but the rate at which \gsw{} declined was
not constant. Thinning reduces parameter estimation bias toward densely
sampled regions of the curve which may not be the most biologically
informative.

\emph{Procedure}: We retained the maxima and minima \gsw{} for each
curve and thinned all but one point per thinning interval of
\(0.05~\log(\si{\mol\raiseto{-2}\meter\raiseto{-1}\second})\), retaining
the point nearest the midpoint of the interval.

\paragraph{\texorpdfstring{Remove replicates with no overlap between
amphi and pseudohypo \agcurve{}
curves}{Remove replicates with no overlap between amphi and pseudohypo  curves}}\label{remove-replicates-with-no-overlap-between-amphi-and-pseudohypo-curves}

\emph{Rationale}: We could not estimate \aax{} for replicates where
amphi and pseudohypo \agcurve{} curves did not overlap.

\emph{Procedure}: We removed replicates where the range of \gsw{} values
for amphi and pseudohypo \agcurve{} curves did not overlap.

\paragraph{\texorpdfstring{Trim extreme \aax{}
values}{Trim extreme  values}}\label{trim-extreme-values}

\emph{Rationale}: Extreme \aax{} values were likely due to measurement
error or leaf damage. Since amphi and pseudohypo \agcurve{} curves are
measured on consecutive days, a poor calibration or a damaged leaf could
cause a large difference in \(A\) between days, which would appear as an
extreme \aax{} value.

\emph{Procedure}: We provionsally estimated \aax{} for each replicate by
integrating over the range of \gsw{} values where amphi and pseudohypo
\agcurve{} curves overlap. In this procedure, curve parameters were
provisionally estimated using ordinary least squares with the
\texttt{lm()} function in \emph{R}. We then used point estimates of
\aax{} for each replicate as the response variable in a linear model
with light treatment, light intensity, accession, and all interactions
as explanatory variables. This model was also fit using ordinary least
squares with the \texttt{lm()} function in \emph{R}. We classified
extreme \aax{} values as those with an absolute internal studentized
residual \(> 3.5\). Because these values likely indicate significant
measurement error or leaf damage, we removed \agcurve{} curves at both
light intensities if either was classified as extreme.

\subsubsection{\texorpdfstring{Cleaning \aqcurve{}
curves}{Cleaning  curves}}\label{cleaning-curves-1}

The raw data set consisted of 658 \aqcurve{} curves with an average of
19.4. Manual curation of a data set this size in a principled,
consistent manner is not feasible. Therefore, we automated data cleaning
using custom \emph{R} scripts. Cleaning is divided into two sequential
steps \autoref{tbl-cleaning2}.

\begin{table}

\caption{\label{tbl-cleaning2}Two sequential steps for cleaning
\agcurve{} curves. The rationale and procedure for each step are
described in the text. The rightmost columns summarize the number of
curves and mean number of points per curve remaining after each step.}

\centering{

\centering
\begin{tabular}{>{\raggedright\arraybackslash}p{9cm}>{\centering\arraybackslash}p{2.5cm}>{\centering\arraybackslash}p{2.5cm}}
\toprule
Step: description & Number of curves & Number of points per curve\\
\midrule
\cellcolor{gray!10}{1. remove outliers within each \aqcurve{} curve} & \cellcolor{gray!10}{658} & \cellcolor{gray!10}{19.1}\\
2. remove \aqcurve{} curves with poor fit & 652 & 19.1\\
\bottomrule
\end{tabular}

}

\end{table}%

\paragraph{\texorpdfstring{Remove outliers within each \aqcurve{}
curve}{Remove outliers within each  curve}}\label{remove-outliers-within-each-curve-1}

\emph{Rationale}: Individual outliers within \agcurve{} curves, usually
caused by transitory changes in chamber conditions, exert undue leverage
on parameter estimates and cause bias and/or low precision in parameter
estimates.

\emph{Procedure}: We fit provisional nonrectangular hyperbola (\emph{8})
to each \aqcurve{} curve using nonlinear regression with the
\texttt{nlsLM()} function from the \emph{R} package \textbf{minpack.lm}
version 1.2.4 (\emph{9}). We sequentially removed data points with an
absolute external studentized residual \(> 3\) until none remained.

\paragraph{\texorpdfstring{Remove \aqcurve{} curves with poor
fit}{Remove  curves with poor fit}}\label{remove-curves-with-poor-fit}

\emph{Rationale}: \aqcurve{} curves with a poor fit to the
nonrectangular hyperbola most likely indicate systematic measurement
error and/or the leaf was not fully acclimated to the chamber
environment.

\emph{Procedure}: As described above, we fit provisional nonrectangular
hyperbola to each \aqcurve{} curve and calculated the model \(r^2\).
There was a clear break between typical curves and poorly fitting curves
where \(r^2 < 0.99\). We therefore removed \aqcurve{} curves with
\(r^2 < 0.99\).

\subsubsection{Data analysis}\label{data-analysis}

\paragraph{General points (not sure where this goes
yet)}\label{general-points-not-sure-where-this-goes-yet}

\begin{itemize}
\tightlist
\item
  models fit in Stan/brms
\item
  we chose the number of chains, warmup and sampling iterations, and
  maximum treedepth so that parameter estimates converged
  (\(\hat{R} < 1.01\) (\emph{10})) and the effective sample size (ESS)
  for each parameter was \(> 10^3\).
\end{itemize}

We fit Bayesian models with MCMC sampling in the probabilistic
programming language \emph{Stan} (\emph{11}). We used CmdStan version
2.33.1 and \textbf{cmdstanr} version 0.7.1 (\emph{12}) to interface with
\emph{R} version 4.3.2 (\emph{13}). We sampled the posterior
distribution from 4 chains with 1000 iterations each after 1000 warmup
iterations per chain. We estimated parameters and confidence intervals
as the median and 95\% quantile intervals of the posterior,
respectively.

\paragraph{Parameter estimation}\label{parameter-estimation}

There were four levels of parameter estimation in our analysis:

\begin{enumerate}
\def\labelenumi{\arabic{enumi}.}
\tightlist
\item
  Estimate \agcurve{} curve parameters
\item
  Estimate \aax{} for each light intensity with leaf using \agcurve{}
  curve parameters
\item
  Estimate the effects of light intensity, light treatment, and
  accession on \aax{} (assimilatory and plasticity hypotheses)
\item
  Estimate the effects of native light habitat on accession-level \aax{}
  (constitutive hypothesis)
\end{enumerate}

Although the higher-level parameter estimates depend on the lower-level
parameter estimates, we fit all models simultaneously to ensure that the
uncertainty in lower-level estimates propagated to higher-level
estimates.

\begin{longtable}{>{\raggedright\arraybackslash}p{1in}>{\raggedright\arraybackslash}p{5in}}

\caption{\label{tbl-parameters}Description of parameters estimated in
the hierarchical Bayesian model. The \texttt{Parameter} column lists the
parameter name as it appears in text. The \texttt{Description} column
provides a brief description of the parameter.}

\tabularnewline

\toprule
Parameter & Description\\
\midrule
\addlinespace[0.3em]
\multicolumn{2}{l}{\textbf{\agcurve{} curve parameters}}\\
\hspace{1em}\cellcolor{gray!10}{$\mathbf{B}_\text{curve}$} & \cellcolor{gray!10}{$n_\text{curve} \times 3$ array of random \agcurve{} curve-level coefficients ($b_{0,j}$, $b_{1,j}$, $b_{2,j}$); $\mathbf{B}_\text{curve} \sim \text{MVN}(\vec{0}, \symbf{\Sigma}_\text{curve})$}\\
\hspace{1em}$\vec{\beta}_\text{curve}$ & vector of mean quadratic coefficients ($\beta_0$, $\beta_1$, $\beta_2$)\\
\hspace{1em}\cellcolor{gray!10}{$\symbf{\Sigma}_\text{curve}$} & \cellcolor{gray!10}{$3 \times 3$ covariance matrix of curve-level coefficients}\\
\hspace{1em}$\sigma_{\qty{6}{\raiseto{2}\centi\meter},\epsilon}$ & minimum residual standard deviation when the measured leaf surface area is $\qty{6}{\raiseto{2}\centi\meter}$\\
\hspace{1em}\cellcolor{gray!10}{$\beta_{S,\epsilon}$} & \cellcolor{gray!10}{slope of the relationship between residual standard deviation and measured leaf surface area (log-link scale)}\\
\hspace{1em}$\rho_{\epsilon}$ & lag-1 residual autocorrelation\\
\addlinespace[0.3em]
\multicolumn{2}{l}{\textbf{\aax{} for each light intensity with leaf using agcurve{} curve parameters}}\\
\hspace{1em}\cellcolor{gray!10}{$\widehat{\mathrm{AA}}_{klm}$} & \cellcolor{gray!10}{estimate of \aax{} for the $k^{\text{th}}$ leaf at light intensity $l$ in accession $m$}\\
\addlinespace[0.3em]
\multicolumn{2}{l}{\textbf{effects light intensity, light treatments, and accession on \aax}}\\
\hspace{1em}$\beta_{\mathrm{AA},0}$ & intercept of \aax{} at low light intensity in shade treatment\\
\hspace{1em}\cellcolor{gray!10}{$\beta_{\mathrm{AA},2000}$} & \cellcolor{gray!10}{effect of high light intensity at \ppfdequals{2000} on \aax}\\
\hspace{1em}$\beta_{\mathrm{AA},\text{sun}}$ & effect of sun treatment on \aax{}\\
\hspace{1em}\cellcolor{gray!10}{$\beta_{\mathrm{AA},\text{2000,high}}$} & \cellcolor{gray!10}{effect of high light intensity at \ppfdequals{2000} on \aax{} in sun treatment}\\
\hspace{1em}$\vec{\beta}_{\mathrm{AA},\text{acc}}$ & vector of $n_\text{acc}$ phylogenetically structured random accession-level effects on \aax; $\vec{\beta}_{\mathrm{AA},\text{acc}} \sim \text{MVN}(\vec{0}, \symbf{\Sigma}_{\text{AA,acc}})$\\
\hspace{1em}\cellcolor{gray!10}{$\symbf{\Sigma}_{\text{AA,acc}}$} & \cellcolor{gray!10}{$n_\text{acc} \times n_\text{acc}$ covariance matrix of phylogenetically structured random accession-level effects on \aax;}\\
\hspace{1em}$\vec{\beta}_{\mathrm{AA},\text{rep}}$ & vector of $n_\text{rep}$ random replicate-level effects on \aax; $\vec{\beta}_{\mathrm{AA},\text{rep}} \sim \text{Normal}(0, \sigma_{\mathrm{AA},\text{rep}})$\\
\hspace{1em}\cellcolor{gray!10}{$\vec{\beta}_{\mathrm{AA},\text{2000,acc}}$} & \cellcolor{gray!10}{vector of $n_\text{acc}$ phylogenetically structured random accession-level effects of high light intensity at \ppfdequals{2000} on \aax; $\vec{\beta}_{\mathrm{AA},\text{2000,acc}} \sim \text{MVN}(\vec{0}, \symbf{\Sigma}_{\mathrm{AA},\text{2000,acc}})$}\\
\hspace{1em}$\vec{\beta}_{\mathrm{AA},\text{sun,acc}}$ & vector of $n_\text{acc}$ phylogenetically structured random accession-level effects of sun treatment on \aax; $\vec{\beta}_{\mathrm{AA},\text{sun,acc}} \sim \text{MVN}(\vec{0}, \symbf{\Sigma}_{\mathrm{AA},\text{sun,acc}})$\\
\hspace{1em}\cellcolor{gray!10}{$\sigma_{\mathrm{AA},\epsilon,0}$} & \cellcolor{gray!10}{intercept of phylogenetically unstructured residual standard deviation of \aax}\\
\hspace{1em}$\beta_{\mathrm{AA},\epsilon,2000}$ & effect of high light intensity at \ppfdequals{2000} on phylogenetically unstructured residual standard deviation of \aax{} (log-link scale)\\
\hspace{1em}\cellcolor{gray!10}{$\beta_{\mathrm{AA},\epsilon,\text{sun}}$} & \cellcolor{gray!10}{effect of sun treatment on phylogenetically unstructured residual standard deviation of \aax{} (log-link scale)}\\
\hspace{1em}$\sigma_{\mathrm{AA},\text{rep}}$ & standard deviation of random replicate-level effects on \aax\\
\hspace{1em}\cellcolor{gray!10}{$\alpha_{\mathrm{AA},\text{acc}}$} & \cellcolor{gray!10}{decay rate of phylogenetic covariance in random accession-level effects on \aax}\\
\hspace{1em}$\sigma^2_{\mathrm{AA},\text{acc}}$ & phylogenetic diffusion rate in random accession-level effects on \aax\\
\hspace{1em}\cellcolor{gray!10}{$\alpha_{\mathrm{AA},\text{2000,acc}}$} & \cellcolor{gray!10}{decay rate of phylogenetic covariance in random accession-level effects of high light intensity at \ppfdequals{2000} on \aax}\\
\hspace{1em}$\sigma^2_{\mathrm{AA},\text{2000,acc}}$ & phylogenetic diffusion rate in random accession-level effects of high light intensity at \ppfdequals{2000} on \aax\\
\hspace{1em}\cellcolor{gray!10}{$\alpha_{\mathrm{AA},\text{sun,acc}}$} & \cellcolor{gray!10}{decay rate of phylogenetic covariance in random accession-level effects of sun treatment on \aax}\\
\hspace{1em}$\sigma^2_{\mathrm{AA},\text{sun,acc}}$ & phylogenetic diffusion rate in random accession-level effects of sun treatment on \aax\\
\addlinespace[0.3em]
\multicolumn{2}{l}{\textbf{effects of native light habitat on accession-level \aax}}\\
\hspace{1em}\cellcolor{gray!10}{$\beta_{\mathrm{AA},\text{PPFD},0}$} & \cellcolor{gray!10}{intercept of accession-level \aax{} when native \ppfdequals{0}}\\
\hspace{1em}$\beta_{\mathrm{AA},\text{PPFD},1}$ & slope of native \ppfd{} on accession-level \aax{}\\
\bottomrule

\end{longtable}

\subparagraph{\texorpdfstring{\agcurve{} curve
parameters}{ curve parameters}}\label{curve-parameters}

We modeled \logA as a quadratic function of \loggsw for each leaf using
the following equation:

\[\log(A_{ij}) = (\beta_0 + b_{0,j}) + (\beta_1 + b_{1,j}) \log{g_{\text{sw},i}} + (\beta_2 + b_{2,j}) \log{g_{\text{sw},i}}^2 + \epsilon_{i}\]

where \(\beta_0\), \(\beta_1\), and \(\beta_2\) are the average
intercept, linear, and quadratic coefficients, respectively. We used
diffuse normal priors with mean \(0\) and standard deviation \(10\) on
these parameters. We estimated random effects of curve \(j\) on the
intercept (\(b_{0,j}\)), linear (\(b_{1,j}\)), and quadratic
(\(b_{2,j}\)) coefficients. We assumed that the \(j \times 3\) array of
coefficients was multivariate normal a mean vector of \(\vec{0}\) and
covariance \(\Sigma_\text{curve}\). We used a weakly informative normal
prior with mean \(0\) and standard deviation \(1\) on the
log-transformed standard deviations (i.e.~the diagonal of
\(\Sigma_\text{curve}\)). We used a weakly informative
\(\mathrm{LJK}(2)\) prior on the correlation matrix. The off diagonal
elements of \(\Sigma_\text{curve}\) can be calculated from its diagonal
elements and the correlation matrix.

The residuals \(\epsilon_{i}\) were modeled as a lag-1 autocorrelated
time-series. We further assumed that the residual standard deviation of
the \(j^{\text{th}}\) curve (\(\sigma_{\epsilon,j}\)) was inversely
proportional to the leaf surface area (\(S_j\)) within the chamber:

\[\log (\sigma_{\epsilon,j}) = \log (\sigma_{\qty{6}{\raiseto{2}\centi\meter},\epsilon}) + (6 - S_j) \beta_{S,\epsilon}\]
where \(\sigma_{\qty{6}{\raiseto{2}\centi\meter},\epsilon}\) is the
minimum residual standard deviation when the
\(\qty{6}{\raiseto{2}\centi\meter}\) chamber is completely filled. The
residual standard deviation increases on log-linear scale by
\(\beta_{S,\epsilon}\). We used a weakly informative normal prior with
mean \(-3\) and standard deviation \(5\) on
\(\log (\sigma_{\qty{6}{\raiseto{2}\centi\meter},\epsilon})\) and a
weakly informative normal prior with mean \(0\) and standard deviation
\(1\) on \(\beta_{S,\epsilon}\).

\subparagraph{\texorpdfstring{\aax{} for each light intensity with leaf
using \agcurve{} curve
parameters}{ for each light intensity with leaf using  curve parameters}}\label{for-each-light-intensity-with-leaf-using-curve-parameters}

Within the \(k^{\text{th}}\) leaf, we estimated \aax{} for each light
intensity by integrating the difference in \logA between the amphi and
pseudohypo \agcurve{} curves over the range of \gsw{} values where the
curves overlap (from \(\min(\log(g_\text{sw}))\) to
\(\max(\log(g_\text{sw}))\)). The estimate of \aax{} for the
\(k^{\text{th}}\) leaf at light intensity \(l\) in accession \(m\) is:

\[\widehat{\mathrm{AA}}_{klm} = \int_{\text{min(log(}g_\text{sw}))}^{\text{max(log(}g_\text{sw}))} \text{log}\bigg(\frac{\hat{A}_\text{amphi}(x; \theta_{klm,\text{amphi})}}{\hat{A}_\text{hypo}(x; \theta_{klm,\text{hypo})}}\bigg) dx\]
where:

\[\theta_\text{amphi} \in \{\hat{b}_{0, f(\text{amphi}, k,l,m)}, \hat{b}_{1, f(\text{amphi}, k,l,m)}, \hat{b}_{2, f(\text{amphi}, k,l,m)}\}, \text{and}\]

\[\theta_\text{hypo} \in \{\hat{b}_{0, f(\text{hypo}, k,l,m)}, \hat{b}_{1, f(\text{hypo}, k,l,m)}, \hat{b}_{2, f(\text{hypo}, k,l,m)}\}.\]
The function \(f : \Theta_1 \rightarrow \Theta_2\) maps the set
\(\Theta_1\) indexed by leaf type (amphi or pseudohypo), leaf replicate,
light intensity, and accession to set \(\Theta_2\) indexed by individual
\agcurve{} curve. This mapping is necessary because the random effects
structure differs between models of \loggsw on \logA and that of models
predicting \aax{} described in the next section.

\subparagraph{\texorpdfstring{Effects of light intensity, light
treatment, and accession on
\aax}{Effects of light intensity, light treatment, and accession on }}\label{effects-of-light-intensity-light-treatment-and-accession-on}

We tested for effects of light intensity, light treatment, accession,
and their interactions on \aax. All models included effects of light
intensity and light treatment, as well as random effects of accession
and replicate within accession. More complex models included
interactions between light intensity and light treatment, as well as
random effects of accession on the effects of light intensity and light
treatment. We used a weakly informative normal prior with mean \(0\) and
standard deviation \(10\) for fixed effects light intensity and
treatment. We used a weakly informative normal prior with mean \(-3\)
and standard deviation \(5\) for the the random effect standard
deviation of replicate within accession. We accounted for the
phylogenetic structure among the random effects of accession using an
Ornstein-Uhlenbeck (OU) process (\textbf{hansen\_stabilizing\_1997?}).
The expected covariance between accessions \(i\) and \(j\)
(\(\text{Cov}(i, j)\) ) is:

\[\text{Cov}(i, j) = \frac{\sigma^2}{2 \alpha} \exp(-\alpha D_{ij})\]
where \(\sigma^2\) is the variance of the random effect, \(\alpha\) is
the rate of decay of the covariance with phylogenetic distance,
\(D_{ij}\), the phylogenetic distance between accessions \(i\) and
\(j\). We estimated \(\sigma^2 / (2 \alpha)\) and \(\alpha\) as a
separate parameters and reparameterized them as \(\sigma^2\) and
\(\alpha\). We used a weakly informative normal prior with mean \(0\)
and standard deviation \(10\) on OU parameters.

We modeled the residual standard deviation of \aax,
(i.e.~phylogenetically unstructured variation unaccounted for by
explanatory variables) on a log-link scale with effects of light
intensity and light treatment. We used a weakly informative normal prior
with mean \(-3\) and standard deviation \(5\) on the residual standard
deviation intercept and a weakly informative normal prior with mean
\(0\) and standard deviation \(1\) on the effects of light intensity and
light treatment on the residual standard deviation.

\subparagraph{\texorpdfstring{Effects of native light habitat on
accession-level
\aax}{Effects of native light habitat on accession-level }}\label{effects-of-native-light-habitat-on-accession-level}

We tested a linear effect of native light habitat on accession-level
\aax. In Models 1 and 2, we used the random intercept of \aax{} as a
response variable; in model 3 we used the random intercept plus the
random effect of accession at high light intensity; in model 4 we used
the random intercept plus the random effect of accession in the sun
treatment; in model 5 we used the random intercept plus the random
effects of accession at high light intensity and sun treatment. We used
a weakly informative normal prior with mean \(0\) and standard deviation
\(1\) for the slope and intercept. We accounted for the phylogenetic
structure among the model residuals using an (OU) process. We used a
weakly informative normal prior with mean \(0\) and standard deviation
\(10\) on OU parameters.

WORKING HERE - TABLE OR SOMETHING SUMMARIZING MODELS? - TABLE
SUMMARIZING PARAMETERS

We estimated these parameters for each leaf using the posterior
distribution of the parameters from the \agcurve{} curve fitting
procedure.

\paragraph{Predictions}\label{predictions}

\begin{itemize}
\tightlist
\item
  Assimilatory:

  \begin{itemize}
  \tightlist
  \item
    average AA2000 - AA150
  \item
    among accession variation in AA2000 - AA150
  \end{itemize}
\item
  Plastic

  \begin{itemize}
  \tightlist
  \item
    average AAsun - AAshade
  \item
    among accession variation in AAsun - AAshade
  \end{itemize}
\item
  Constitutive

  \begin{itemize}
  \tightlist
  \item
    cor(PAR, AA) at accession level
  \end{itemize}
\end{itemize}

Reinforcing interactions (we only consider these because these are ones
which could explain why amphi leaves occur in high light habitats)

\begin{itemize}
\tightlist
\item
  Assimilatory * Plastic (reinforcing interaction)

  \begin{itemize}
  \tightlist
  \item
    AAsun - AAshade greater at 2000 than 150
  \item
    AA2000 - AA150 greater in sun leaves than shade leaves
  \item
    there may be var among accessions, but not correlated with PAR
  \end{itemize}
\item
  Assimilatory * Constitutive (reinforcing interaction)

  \begin{itemize}
  \tightlist
  \item
    cor(AA2000, PAR) \textgreater{} 0; cor(AA150, PAR) = 0
  \item
    AA2000 - AA150 varies among accessions, correlated with PAR
  \end{itemize}
\item
  Plastic * Constitutive (reinforcing interaction)

  \begin{itemize}
  \tightlist
  \item
    cor(AAsun, PAR) \textgreater{} 0; cor(AAshade, PAR) = 0
  \item
    AAsun - AAshade varies among accessions, correlated with PAR
  \end{itemize}
\item
  Assimilatory * Plastic * Constitutive (reinforcing interaction)

  \begin{itemize}
  \tightlist
  \item
    AAsun - AAshade greater at 2000 than 150
  \item
    AA2000 - AA150 greater in sun leaves than shade leaves
  \item
    AAsun - AAshade varies among accession, correlated with PAR
  \item
    AA2000 - AA150 varies among accession, correlated with PAR
  \item
    cor(AAsun2000, PAR) \textgreater{} cor(AAsun150, PAR) and
    \textgreater{} cor(AAshade2000, PAR) and \textgreater{}
    cor(AAshade150, PAR) in nonadditive way
  \end{itemize}
\end{itemize}

We tested competing predictions through a combination of model selection
and parameter estimation from the posterior of selected models. We
started with a base model that includes fixed effects of light intensity
and light treatment on \aax, a random effect of accession on \aax, and a
linear regression of PAR on accession-level \aax. We next compared the
fit of the base model to a more complex with an interaction between
light intensity and light treatment using LOOIC with the \textit{R}
package \textbf{loo} version 2.7.0 (\emph{14}). These comparisons
allowed us to test the main predictions of each hypothesis and for a
reinforcing interactions between assimilatory and plastic hypotheses
(tab:predictions). However, these models cannot test predictions when
there is an interaction with the constitutive hypothesis, because this
hypothesis posits accession-levels variation in responses to light
intensity and/or light treatment. Therefore, we tested for significant
variation among accessions in light intensity and light treatment by
comparing LOOIC values of models with and without random interactions
between these factors and accession.

\begin{longtable}[]{@{}
  >{\raggedright\arraybackslash}p{(\columnwidth - 4\tabcolsep) * \real{0.3182}}
  >{\raggedright\arraybackslash}p{(\columnwidth - 4\tabcolsep) * \real{0.2727}}
  >{\raggedright\arraybackslash}p{(\columnwidth - 4\tabcolsep) * \real{0.4091}}@{}}
\toprule\noalign{}
\begin{minipage}[b]{\linewidth}\raggedright
Hypothesis
\end{minipage} & \begin{minipage}[b]{\linewidth}\raggedright
Prediction
\end{minipage} & \begin{minipage}[b]{\linewidth}\raggedright
Model
\end{minipage} \\
\midrule\noalign{}
\endhead
\bottomrule\noalign{}
\endlastfoot
Assimilatory & & base model \\
Plastic & & base model \\
Constitutive & & base model \\
Assimilatory * Plastic & & + intensity \(\times\) treatment \\
Assimilatory * Constitutive & & + accession \(\times\) intensity
(random) \\
Plastic * Constitutive & & + accession \(\times\) treatment (random) \\
Assimilatory * Plastic * Constitutive & & + accession \(\times\)
intensity \(\times\) treatment (random) \\
\end{longtable}

\subsection{Acknowledgements}\label{acknowledgements}

Justin Alter, Max Gatlin, Joana Kim, Jenna Matsuyama, Brandon Najarian,
Kai Yasuda

\subsection*{References}\label{references}
\addcontentsline{toc}{subsection}{References}

\phantomsection\label{refs}
\begin{CSLReferences}{0}{1}
\bibitem[\citeproctext]{ref-triplett_amphistomy_2024}
\CSLLeftMargin{1. }%
\CSLRightInline{G. Triplett, T. N. Buckley, C. D. Muir,
\href{https://doi.org/10.1002/ajb2.16284}{Amphistomy increases leaf
photosynthesis more in coastal than montane plants of {Hawaiian} ʻilima
( \emph{{Sida} fallax} )}. \emph{American Journal of Botany}
\textbf{111}, e16284 (2024).}

\bibitem[\citeproctext]{ref-peralta_taxonomy_2008}
\CSLLeftMargin{2. }%
\CSLRightInline{I. E. Peralta, D. M. Spooner, S. Knapp, Taxonomy of wild
tomatoes and their relatives (\emph{solanum} sect.
\emph{Lycopersicoides}, sect. \emph{Juglandifolia}, sect.
\emph{Lycopersicon}; {Solanaceae}). \textbf{84} (2008).}

\bibitem[\citeproctext]{ref-schoch_dependence_1980}
\CSLLeftMargin{3. }%
\CSLRightInline{P.-G. Schoch, C. Zinsou, M. Sibi,
\href{https://doi.org/10.1093/jxb/31.5.1211}{Dependence of the stomatal
index on environmental factors during stomatal differentiation in leaves
of \emph{{Vigna} sinensis} {L}.: 1. {Effect} of light intensity}.
\emph{Journal of Experimental Botany} \textbf{31}, 1211--1216 (1980).}

\bibitem[\citeproctext]{ref-parkhurst_adaptive_1978}
\CSLLeftMargin{4. }%
\CSLRightInline{D. F. Parkhurst,
\href{https://doi.org/10.2307/2259142}{The adaptive significance of
stomatal occurrence on one or both surfaces of leaves}. \emph{The
Journal of Ecology} \textbf{66}, 367--383 (1978).}

\bibitem[\citeproctext]{ref-sack_developmental_2016}
\CSLLeftMargin{5. }%
\CSLRightInline{L. Sack, T. N. Buckley,
\href{https://doi.org/10.1104/pp.16.00476}{The developmental basis of
stomatal density and flux}. \emph{Plant Physiology} \textbf{171},
2358--2363 (2016).}

\bibitem[\citeproctext]{ref-mott_amphistomy_1991}
\CSLLeftMargin{6. }%
\CSLRightInline{K. A. Mott, O. Michaelson, Amphistomy as an adaptation
to high light intensity in \emph{{Ambrosia} cordifolia} ({Compositae}).
\emph{American Journal of Botany} \textbf{78}, 76--79 (1991).}

\bibitem[\citeproctext]{ref-schindelin_fiji_2012}
\CSLLeftMargin{7. }%
\CSLRightInline{J. Schindelin, I. Arganda-Carreras, E. Frise, V. Kaynig,
M. Longair, T. Pietzsch, S. Preibisch, C. Rueden, S. Saalfeld, B.
Schmid, J.-Y. Tinevez, D. J. White, V. Hartenstein, K. Eliceiri, P.
Tomancak, A. Cardona, \href{https://doi.org/10.1038/nmeth.2019}{Fiji: An
open-source platform for biological-image analysis}. \emph{Nature
Methods} \textbf{9}, 676--682 (2012).}

\bibitem[\citeproctext]{ref-marshall_model_1980}
\CSLLeftMargin{8. }%
\CSLRightInline{B. Marshall, P. V. Biscoe,
\href{https://doi.org/10.1093/jxb/31.1.29}{A {Model} for
{C}\(_{\textrm{3}}\) {Leaves} {Describing} the {Dependence} of {Net}
{Photosynthesis} on {Irradiance}}. \emph{Journal of Experimental Botany}
\textbf{31}, 29--39 (1980).}

\bibitem[\citeproctext]{ref-elzhov_minpacklm_2023}
\CSLLeftMargin{9. }%
\CSLRightInline{T. V. Elzhov, K. M. Mullen, A.-N. Spiess, B. M. Bolker,
\emph{Minpack.lm: {R} {Interface} to the {Levenberg}-{Marquardt}
{Nonlinear} {Least}-{Squares} {Algorithm} {Found} in {MINPACK}, {Plus}
{Support} for {Bounds}} (2023;
\url{https://CRAN.R-project.org/package=minpack.lm}).}

\bibitem[\citeproctext]{ref-gelman_inference_1992}
\CSLLeftMargin{10. }%
\CSLRightInline{A. Gelman, D. B. Rubin, Inference from iterative
simulation using multiple sequences. \emph{Statistical Science}
\textbf{7}, 457--472 (1992).}

\bibitem[\citeproctext]{ref-stan_development_team_stan_2023}
\CSLLeftMargin{11. }%
\CSLRightInline{Stan Development Team, \emph{Stan {Modeling} {Language}
{Users} {Guide} and {Reference} {Manual}} (2023;
\url{https://mc-stan.org}).}

\bibitem[\citeproctext]{ref-gabry_cmdstanr_2023}
\CSLLeftMargin{12. }%
\CSLRightInline{J. Gabry, R. Češnovar, \emph{Cmdstanr: {R} {Interface}
to '{CmdStan}'} (2023;
\href{https://mc-stan.org/cmdstanr,\%20https://discourse.mc-stan.org}{https://mc-stan.org/cmdstanr,
https://discourse.mc-stan.org}).}

\bibitem[\citeproctext]{ref-r_core_team_r:_2023}
\CSLLeftMargin{13. }%
\CSLRightInline{R Core Team, \emph{R: {A} {Language} and {Environment}
for {Statistical} {Computing}} (R Foundation for Statistical Computing,
Vienna, Austria, 2023; \url{http://www.R-project.org/}).}

\bibitem[\citeproctext]{ref-vehtari_practical_2017}
\CSLLeftMargin{14. }%
\CSLRightInline{A. Vehtari, A. Gelman, J. Gabry,
\href{https://doi.org/10.1007/s11222-016-9696-4}{Practical {Bayesian}
model evaluation using leave-one-out cross-validation and {WAIC}}.
\emph{Statistics and Computing} \textbf{27}, 1413--1432 (2017).}

\end{CSLReferences}



\end{document}
