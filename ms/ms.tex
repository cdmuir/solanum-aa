% Options for packages loaded elsewhere
\PassOptionsToPackage{unicode}{hyperref}
\PassOptionsToPackage{hyphens}{url}
\PassOptionsToPackage{dvipsnames,svgnames,x11names}{xcolor}
%
\documentclass[
  letterpaper,
  DIV=11,
  numbers=noendperiod]{scrartcl}

\usepackage{amsmath,amssymb}
\usepackage{iftex}
\ifPDFTeX
  \usepackage[T1]{fontenc}
  \usepackage[utf8]{inputenc}
  \usepackage{textcomp} % provide euro and other symbols
\else % if luatex or xetex
  \usepackage{unicode-math}
  \defaultfontfeatures{Scale=MatchLowercase}
  \defaultfontfeatures[\rmfamily]{Ligatures=TeX,Scale=1}
\fi
\usepackage{lmodern}
\ifPDFTeX\else  
    % xetex/luatex font selection
  \setmainfont[]{Times New Roman}
\fi
% Use upquote if available, for straight quotes in verbatim environments
\IfFileExists{upquote.sty}{\usepackage{upquote}}{}
\IfFileExists{microtype.sty}{% use microtype if available
  \usepackage[]{microtype}
  \UseMicrotypeSet[protrusion]{basicmath} % disable protrusion for tt fonts
}{}
\makeatletter
\@ifundefined{KOMAClassName}{% if non-KOMA class
  \IfFileExists{parskip.sty}{%
    \usepackage{parskip}
  }{% else
    \setlength{\parindent}{0pt}
    \setlength{\parskip}{6pt plus 2pt minus 1pt}}
}{% if KOMA class
  \KOMAoptions{parskip=half}}
\makeatother
\usepackage{xcolor}
\setlength{\emergencystretch}{3em} % prevent overfull lines
\setcounter{secnumdepth}{5}
% Make \paragraph and \subparagraph free-standing
\ifx\paragraph\undefined\else
  \let\oldparagraph\paragraph
  \renewcommand{\paragraph}[1]{\oldparagraph{#1}\mbox{}}
\fi
\ifx\subparagraph\undefined\else
  \let\oldsubparagraph\subparagraph
  \renewcommand{\subparagraph}[1]{\oldsubparagraph{#1}\mbox{}}
\fi


\providecommand{\tightlist}{%
  \setlength{\itemsep}{0pt}\setlength{\parskip}{0pt}}\usepackage{longtable,booktabs,array}
\usepackage{calc} % for calculating minipage widths
% Correct order of tables after \paragraph or \subparagraph
\usepackage{etoolbox}
\makeatletter
\patchcmd\longtable{\par}{\if@noskipsec\mbox{}\fi\par}{}{}
\makeatother
% Allow footnotes in longtable head/foot
\IfFileExists{footnotehyper.sty}{\usepackage{footnotehyper}}{\usepackage{footnote}}
\makesavenoteenv{longtable}
\usepackage{graphicx}
\makeatletter
\def\maxwidth{\ifdim\Gin@nat@width>\linewidth\linewidth\else\Gin@nat@width\fi}
\def\maxheight{\ifdim\Gin@nat@height>\textheight\textheight\else\Gin@nat@height\fi}
\makeatother
% Scale images if necessary, so that they will not overflow the page
% margins by default, and it is still possible to overwrite the defaults
% using explicit options in \includegraphics[width, height, ...]{}
\setkeys{Gin}{width=\maxwidth,height=\maxheight,keepaspectratio}
% Set default figure placement to htbp
\makeatletter
\def\fps@figure{htbp}
\makeatother
% definitions for citeproc citations
\NewDocumentCommand\citeproctext{}{}
\NewDocumentCommand\citeproc{mm}{%
  \begingroup\def\citeproctext{#2}\cite{#1}\endgroup}
\makeatletter
 % allow citations to break across lines
 \let\@cite@ofmt\@firstofone
 % avoid brackets around text for \cite:
 \def\@biblabel#1{}
 \def\@cite#1#2{{#1\if@tempswa , #2\fi}}
\makeatother
\newlength{\cslhangindent}
\setlength{\cslhangindent}{1.5em}
\newlength{\csllabelwidth}
\setlength{\csllabelwidth}{3em}
\newenvironment{CSLReferences}[2] % #1 hanging-indent, #2 entry-spacing
 {\begin{list}{}{%
  \setlength{\itemindent}{0pt}
  \setlength{\leftmargin}{0pt}
  \setlength{\parsep}{0pt}
  % turn on hanging indent if param 1 is 1
  \ifodd #1
   \setlength{\leftmargin}{\cslhangindent}
   \setlength{\itemindent}{-1\cslhangindent}
  \fi
  % set entry spacing
  \setlength{\itemsep}{#2\baselineskip}}}
 {\end{list}}
\usepackage{calc}
\newcommand{\CSLBlock}[1]{\hfill\break\parbox[t]{\linewidth}{\strut\ignorespaces#1\strut}}
\newcommand{\CSLLeftMargin}[1]{\parbox[t]{\csllabelwidth}{\strut#1\strut}}
\newcommand{\CSLRightInline}[1]{\parbox[t]{\linewidth - \csllabelwidth}{\strut#1\strut}}
\newcommand{\CSLIndent}[1]{\hspace{\cslhangindent}#1}

\usepackage{booktabs}
\usepackage{longtable}
\usepackage{array}
\usepackage{multirow}
\usepackage{wrapfig}
\usepackage{float}
\usepackage{colortbl}
\usepackage{pdflscape}
\usepackage{tabu}
\usepackage{threeparttable}
\usepackage{threeparttablex}
\usepackage[normalem]{ulem}
\usepackage{makecell}
\usepackage{xcolor}
\KOMAoption{captions}{tableheading}
\usepackage{hyperref}
\addtokomafont{disposition}{\rmfamily}
\usepackage{siunitx}
\newcommand{\aax}{$\mathrm{AA}$}
\newcommand{\Aamphi}{$A_{\mathrm{amphi}}$}
\newcommand{\Ahypo}{$A_{\mathrm{hypo}}$}
\newcommand{\agcurve}{$A \textendash g_\text{sw}$}
\newcommand{\aqcurve}{$A \textendash Q$}
\newcommand{\ca}{$C_\mathrm{a}$}
\newcommand{\caequals}[1]{$C_\mathrm{a} = \qty{#1}{\micro\mol\raiseto{-1}\mol}$}
\newcommand{\cabetween}[2]{#1 < $C_\mathrm{a} < \qty{#2}{\micro\mol\raiseto{-1}\mol}$}
\newcommand{\gmaxratio}{$g_\text{max,ratio}$}
\newcommand{\gsw}{$g_\text{sw}$}
\newcommand{\loggsw}{$\log(g_\text{sw})$}
\newcommand{\logA}{$\log(A)$}
\newcommand{\ppfd}{$\mathrm{PPFD}$}
\newcommand{\ppfdequals}[1]{$\mathrm{PPFD} = \qty{#1}{\micro\mol\raiseto{-2}\meter\raiseto{-1}\second}$}
\newcommand{\ppfdqty}[1]{$\qty{#1}{\micro\mol\raiseto{-2}\meter\raiseto{-1}\second}$}
\newcommand{\rh}{$\mathrm{RH}$}
\newcommand{\rhequals}[1]{$\mathrm{RH} = #1\%$}
\newcommand{\tleafequals}[1]{$T_\mathrm{leaf} = \qty{#1}{\degreeCelsius}$}
\makeatletter
\@ifpackageloaded{caption}{}{\usepackage{caption}}
\AtBeginDocument{%
\ifdefined\contentsname
  \renewcommand*\contentsname{Table of contents}
\else
  \newcommand\contentsname{Table of contents}
\fi
\ifdefined\listfigurename
  \renewcommand*\listfigurename{List of Figures}
\else
  \newcommand\listfigurename{List of Figures}
\fi
\ifdefined\listtablename
  \renewcommand*\listtablename{List of Tables}
\else
  \newcommand\listtablename{List of Tables}
\fi
\ifdefined\figurename
  \renewcommand*\figurename{Figure}
\else
  \newcommand\figurename{Figure}
\fi
\ifdefined\tablename
  \renewcommand*\tablename{Table}
\else
  \newcommand\tablename{Table}
\fi
}
\@ifpackageloaded{float}{}{\usepackage{float}}
\floatstyle{ruled}
\@ifundefined{c@chapter}{\newfloat{codelisting}{h}{lop}}{\newfloat{codelisting}{h}{lop}[chapter]}
\floatname{codelisting}{Listing}
\newcommand*\listoflistings{\listof{codelisting}{List of Listings}}
\makeatother
\makeatletter
\makeatother
\makeatletter
\@ifpackageloaded{caption}{}{\usepackage{caption}}
\@ifpackageloaded{subcaption}{}{\usepackage{subcaption}}
\makeatother
\ifLuaTeX
  \usepackage{selnolig}  % disable illegal ligatures
\fi
\usepackage{bookmark}

\IfFileExists{xurl.sty}{\usepackage{xurl}}{} % add URL line breaks if available
\urlstyle{same} % disable monospaced font for URLs
\hypersetup{
  pdftitle={Unknown},
  colorlinks=true,
  linkcolor={blue},
  filecolor={Maroon},
  citecolor={Blue},
  urlcolor={Blue},
  pdfcreator={LaTeX via pandoc}}

\title{Unknown}
\author{}
\date{}

\begin{document}
\maketitle

\renewcommand*\contentsname{Table of contents}
{
\hypersetup{linkcolor=}
\setcounter{tocdepth}{3}
\tableofcontents
}
\section{Notes}\label{notes}

Terminology to standardize:

\begin{itemize}
\item
  the native light habitat or environment: the SPLASH paper refers to
  what they calculated as `habitat' PPFD, so I am going to use that
  terminology
\item
  growth condition versus measurement conditions
\item
  ?
\end{itemize}

\section{Authors}\label{authors}

definite: Wei Shen Lim, Dachuan Wang, Chris Muir

question: Sam McKlin, Tom Buckley

I am using this approach to drafting the paper:

https://blogs.nature.com/nyc/2011/08/10/how-to-write-a-paper-one-possible-answer

\section{Punchline}\label{punchline}

Plastic changes in the structure of sun leaves can increase the
intercellular CO\(_2\) diffusion path length. Gas exchange through
stomata on both upper and lower leaf surfaces, amphistomy, may be a
compensatory mechanism to reduce the impact of increased diffusion path
length on photosynthesis without increasing water loss. We demonstrate
for the first time that sun leaves benefit most from amphistomy because
it compensates for plastic structural changes in leaf anatomy.

\section{Figures}\label{figures}

\begin{enumerate}
\def\labelenumi{\arabic{enumi}.}
\tightlist
\item
  competing hypotheses and predictions (or predictions after methods?)
  for the benefit of amphistomy
\item
  design of experiment to test competing hypotheses (with actual \#'s of
  replicate, curves, etc.)
\item
  Results on acclimatory, plastic, constitutive hypotheses
\item
  Additional supporting figure on change in LMA, reduction of Amass, and
  relationship between LMA and AA? Modeling?
\end{enumerate}

\section{Abstract}\label{abstract}

(use for GRC)

Developmental plasticity to light intensity modulates amphistomy
advantage

The presence of stomata on both leaf surfaces (amphistomy) can increase
photosynthesis in C\(_3\) plants by reducing the path length for
CO\(_2\) diffusion between substomatal cavities and chloroplasts.
Amphistomatous leaves are most common among herbaceous plants growing in
sunny habitats, including many crop species. The distribution of
amphistomatous leaves in nature may result from an increased
photosynthetic benefit of amphistomy under high light intensity, either
because of acclimatory, plastic, or constitutive variation in CO\(_2\)
supply or demand. We used a recently developed method to quantify
amphistomy advantage, the photosynthetic rate of an amphistomatous leaf
relative to an otherwise identical hypostomatous leaf, in 29 diverse
populations representing 15 species of wild tomatoes (\emph{Solanum}
sect. \emph{Lycopersicon} and sect. \emph{Lycopersicoides}). Plant grown
under high light intensity benefit more from amphostomy than those grown
under low light, regardless of acclimation or native light intensity.
Curvature of light response curves indicates that high-light leaves
benefit more from amphistomy because of greater resistance to CO\(_2\)
diffusion within the mesophyll. Developmental plasticity in leaf
thickness and/or cell arrangement to optimize light interception likely
increases under varying light intensity modulates CO\(_2\) diffusion
within the leaf and, hence, amphistomy advantage as a byproduct.
Developmental plasticity may therefore play an important and previously
unnoticed role in explaining the adaptive significance of amphistomy and
the distribution of amphistomatous leaves in nature.

What's surprising: role of developmental plasticity in explaining
comparative results; diffusional limitations through palisade ias is
strong and needs to be considered with light gradients in understanding
leaf strucuture- funciton

\section{Intro}\label{intro}

Stomata are microscopic pores on the surfaces of leaves and other
photosynthetic organs formed by a pair of guard cells. They are
essential for balancing carbon gained per unit water lost and permitted
vascular plants to grow tall on land by enabling access to CO\(_2\) for
photosynthetic carbon assimilation while preventing hydraulic failure in
variable environments (\emph{1}--\emph{3}). Optimal stomatal function
depends on both dynamic changes in aperture on the scale of minutes to
hours, as well as static anatomy determined by developmental plasticity
and constitutive genetic differences (\emph{4}--\emph{6}). Understanding
how stomata respond to environmental change over daily, developmental,
and evolutionary time is important for understanding adaptation
(\emph{1}, \emph{7}--\emph{12}), predicting paleoclimate from fossil
cuticles (\emph{13}--\emph{15}), and improving crops (\emph{16}).
Stomatal function contributes to global carbon and water cycles
(\emph{17}) and therefore predicting future climate (\emph{18}).

Despite extensive theoretical and empirical progress understanding
stomata function and anatomy from molecular to ecosystem levels, the
adaptive significance of amphistomatous leaves remains an important
unsolved problem in leaf structure-function relationships
(\emph{19}--\emph{26}). Amphistomatous leaves develop abaxial and
adaxial stomata whose aperture can be independently regulated
(\emph{27}--\emph{31}) to control gas exchange through each surface. All
else being equal, gas exchange through stomata on both surfaces
increases CO\(_2\) supply to chloroplasts by providing a second parallel
pathway through leaf intercellular airspaces, enhancing photosynthesis
(\emph{20}, \emph{32}). The extent to which amphistomy increases
CO\(_2\) supply depends on resistance to diffusion in intercellular
airspaces. This resistance can be low in thin, porous, amphistomatous
leaves (\emph{28}, \emph{33}), but may be more substantial in thick,
dense, hypostomatous leaves (\emph{34}). Amphistomatous leaves also lose
more water through evaporation because of a second boundary layer
conductance (\emph{35}), but the additional carbon gain should be enough
to offset this cost in most realistic scenarios (\emph{36}).

The paradoxical fact is that, despite the photosynthetic benefit, most
leaves are not amphistomatous. Many vertically oriented and/or
isobilateral leaves are amphistomatous (\emph{25}). But among
dorsiventral leaves, herbaceous plants in open, high light habitats tend
to have have amphistomatous leaves (\emph{22}, \emph{39}--\emph{44}).
Most other leaves, except those from aquatic habitats, are
hypostomatous, producing stomata only on the lower, abaxial surface.
Even resupinate leaves develop stomata on the lower, albeit adaxial
surface (\emph{45}), suggesting that leaf orientation (lower vs.~upper)
rather than leaf polarity (abaxial vs.~adaxial) is causal. The
covariation between stomatal density ratio and light habitat is both
qualitative and quantitative. A higher proportion of sun leaves are
amphistomatous (\textbf{salisbury\_causes\_1928?}) and the proportion of
stomata on the upper, adaxial surface increases with light (\emph{42},
\emph{43}). Resolving why high light intensity favors amphistomatous
dorsivental leaves is an important first step toward understanding
variation in stomatal density ratio and leaf structure-function
relationships more generally.

The overarching hypothesis is that leaves with greater stomatal density
ratio are more common in open, sunny habitats because they increase
photosynthesis most in those circumstances. An amphistomatous leaf
increases photosynthetic carbon gain compared to an otherwise identical
hypostomatous leaf by increasing conductance through the leaf
intercellular airspaces and boudary layers; the additional water loss
through a second boundary layer is typically small (\emph{35}). We
quantify this benefit as the amphistomy advantage (AA =
log(Aamphi/Ahypo) (\emph{20}, \emph{46}). Why would AA be greater in sun
than shade? We consider three nonmutually exclusive hypotheses that we
classify as `acclimatory', `plastic', and `constitutive'.

Acclimatory hypothesis: Photosynthetic induction to high light intensity
typically involves increases in total leaf stomatal conductance
(increased CO\(_2\) supply), the concentration of active Rubisco, and
electron transport capacity (increased CO\(_2\) demand). A
one-dimensional circuit model using the Farquhar-von Caemmerer-Berry
biochemical model of C\(_3\) photosynthesis (\emph{47}) shows that both
increased stomatal conductance and Rubisco activity should increase AA,
all else being equal (Supporting Information). If the acclimatory
hypothesis is correct, we predict that AAhigh \textgreater{} AAlow for
all species regardless of native habitat or growth environment. Plants
adapted to sunny, open habitats will evolve greater stomatal density
ratio to take advantage of regular exposure to high light intensity.

Plastic hypothesis: Individuals of the same genotype often develop
dramatically different leaves in sun and shade conditions (\emph{48}).
Plastic responses are likely adaptations to optimize photosynthesis at
different light intensities in variable environments (\emph{49}).
Plastic changes in leaf anatomy and biochemistry could modulate AA as a
byproduct. Thicker or less porous leaves, both of which are associated
with high leaf mass per area (LMA), will have lower \(g_\mathrm{ias}\);
leaves with increased total stomatal density and photosynthetic capacity
have greater potential CO\(_2\) supply and demand. Under the plastic
hypothesis, we predict that AAsun \textgreater{} AAshade for all species
and light intensities. Secondarily, AAsun and gsmax,sun should also be
positively associated with native light habitat if genotypes adapted to
sunny, open habitats if they can express a phenotype best adapted to
that environment when leaves develop under high light intensity.
Genotypes adapted to shaded, closed habitats may be plastic, but limits
on the width of their reaction norms prevent them from developing traits
optimal for conditions they do not regularly experience in nature.

Constitutive hypothesis: In environments that are relatively constant or
where environmental change cannot be anticipated by a reliable cue,
natural selection will favor constitutive expression of optimal
phenotypes. We therefore predict genotypes from more sunny, open
habitats will have consistently greater AA under also measurement and
growth light intensities. For herbaceous plants, light intensity is
largely a function of the tree canopy (\emph{50}). Herbs growing in the
open will regularly experience high light intensity; herbs growing under
a forest canopy will often experience low light intensity.

The primary directional predictions for each hypothesis are summarized
in Table X; detailed predictions for results that would indicate support
for multiple hypotheses are in Table SX.

CONCEPTUAL FIGURE HERE

caption: Directional predictions associated with each hypothesis
explaining why amphistomy advantage (AA) might be greater for leaves in
sunny, open habitats. For each hypothesis, we make predictions for how
native plant area index (PAI), growth light treatment, and measurement
light intensity would affect AA.

\begin{longtable}[]{@{}
  >{\centering\arraybackslash}p{(\columnwidth - 6\tabcolsep) * \real{0.1728}}
  >{\centering\arraybackslash}p{(\columnwidth - 6\tabcolsep) * \real{0.3580}}
  >{\centering\arraybackslash}p{(\columnwidth - 6\tabcolsep) * \real{0.3210}}
  >{\centering\arraybackslash}p{(\columnwidth - 6\tabcolsep) * \real{0.1481}}@{}}
\toprule\noalign{}
\begin{minipage}[b]{\linewidth}\centering
hypothesis
\end{minipage} & \begin{minipage}[b]{\linewidth}\centering
measurement light intensity
\end{minipage} & \begin{minipage}[b]{\linewidth}\centering
growth light intensity
\end{minipage} & \begin{minipage}[b]{\linewidth}\centering
native PAI
\end{minipage} \\
\midrule\noalign{}
\endhead
\bottomrule\noalign{}
\endlastfoot
acclimatory & AA2000 \textgreater{} AA150 & AAsun = AAshade & cor(PAI,
AA) = 0 \\
plastic & AA2000 = AA150 & AAsun \textgreater{} AAshade & cor(PAI, AA) =
0 \\
constitutive & AA2000 = AA150 & AAsun = AAshade & cor(PAI, AA)
\textgreater{} 0 \\
\end{longtable}

We tested these hypotheses by comparing AA among amphistomatous wild
tomato species (\emph{51}) from different native light habitats, grown
under simulated sun and shade light treatments, and measured under
contrasting light intensity (Figure of hypotheses and predictions). We
measured AA on 572 individual plants from 29 accessions (average of 9.86
replicates per light treatment) using a recently developed method
(\emph{46}). With this method, we directly compare the photosynthetic
rate of an untreated amphistomatous leaf to that of the same leaf with
gas exchange blocked through the adaxial (upper) surface by transparent
plastic, which we refer to as `pseudohypostomy'. To compare amphi- and
pseudohypostomatous leaves at identical whole-leaf \(g_\text{sc}\), we
measure \(A\) over a range of \(g_\text{sc}\), inducing stomatal opening
and closure by modulating humidity (see Materials and Methods for
further details). We estimated `amphistomy advantage' (\aax)
\emph{sensu} (\emph{20}), but with modifications previously described in
(\emph{46}) and here (Materials and Methods). The native light intensity
was represented by plant area index (PAI
\(\unit{\meter\squared\per\meter\squared}\)), estimated using a global
gridded X-m2 resolution data set derived from the Global Ecosystem
Dynamics Investigation {[}GEDI; (\emph{52}){]} and georeferenced
accession collection information from the Tomato Genetics Resource
Center. The growth light intensities were 761 (sun treatment) and 115
(shade treatment) while all other environment conditions were nearly
identical (see \textbf{?@sec-methods} for supporting detail). The high
and low measurements intensities were \ppfdequals{2000} (97.8:2.24
red:blue) and \ppfdequals{150} (87.0:13.0 red:blue), respectively.

\subsubsection{MARKER}\label{marker}

Consistent with biophysical theory of CO\(_2\) diffusion within leaves,
AA \textgreater{} 0 for all accessions (Figure X). AA varied
substantially between measurement light intensities, growth light
intensities, and among accessions (point to model comparison evidence).
Measured under high light intensity, AA was consistently greater for sun
plants. The average AA among accessions in the shade treatment was 0.039
(range: 0.012--0.109; 21 of 29 accessions significant); however, the
same accessions grown at high light intensity showed a mean AA of 0.050
(range: 0.024--0.120; 25 of 29 accessions significant). Contrary to the
predictions of the assimilatory hypothesis, AA was greater in all
accessions under low measurement light intensity for both sun and shade
grown plants. The effect of low light on AA was more pronounced in the
sun-grown plants, where AA was significantly greater under low
measurement light intensity in 28 of 29 accessions compared to to 11
accessions for shade-grown plants. The overall average AA of shade and
sun grown plants measured under low light intensity was 0.066 (range:
0.019--0.143; 27 of 29 accessions significant) and 0.098 (range:
0.051--0.175; 29 of 29 accessions significant), respectively. There was
a slight tendency for accessions from more closed habitats (greater
PAI), but AA varied widely among accessions from open habitats (low PAI)
regardless of growth and measurement light intensities (Figure X). The
pattern of AA across wild tomatoes strongly supports the plasticity
hypothesis, argues against the acclimatory hypothesis, and provides only
weak support for the constitutive hypothesis.

Not sure where this goes: We infer this from the fact that blocking gas
exchange in pseudohypostomatous leaves reduced \(A\) by X-X\% depending
on the accession, light treatment, and light intensity (Table/figure).
The AA is equivalent to an X-X\% change in total \(g_\text{sc}\) (see SI
section gs equivalency). But whereas increasing \(g_\text{sc}\) would
increase water loss as a necessary by-product, amphistomy can increase
\(A\) without any appreciable affect on transpiration.

These results suggest that developmental plasticity, as opposed to
acclimation or adaptation to open habitats, may explain the explain the
long-standing observation taht amphistomatous leaves are more common in
sunny habitats. Our results add to existing explanations by showing high
light intensity \emph{per se} does not increase the benefit of
amphistomy, but rather than anatomical and biochemical change associated
with higher light intensity modulate AA. We cannot yet attribute
specific changes wrought by plasticity, but correlational evidence
suggests that changes in leaf anatomy may be important. Species
responded consistently by increasing stomatal density, especially on the
adaxial surface, increasing stomatal density ratio. Sun leaves also had
greater gs, Amax. However, these traits were not associated with change
in AA (table/figures).

from Mott and Oleary: ``However, Jones and Slatyer (5) reported a higher
mesophyll resistance for CO, entering through the upper stomata than for
the lower, and the data of Vaclavik (12) appear to support this
conclusion''

\section{Outtro - I think this goes at the
end??}\label{outtro---i-think-this-goes-at-the-end}

It is commonly assumed in comparative studies that most trait variation
between species is constitutive and genetically determined, but our
results suggest that plastic responses to light are necessary to explain
why \gmaxratio{} increases with light habitat. Reaction norms to light
intensity among wild tomatoes indicate that accessions from low light
habitats express low \gmaxratio{} when grown in shade, whereas
accessions from high light habitats have constitutively higher
\gmaxratio{} (Fig. rxn-norm). The larger response to light intensity is
driven by greater plasticty in leaf structure in accessions from
low-light habitats (Fig/table - I haven't tested this yet, so this is a
placeholder.) We infer that these structural changes in sun leaves are
costly because they increase the diffusion path length for CO\(_2\) and
reduce photosynthesis. As a consequence, sun leaves benefit more from
amphistomy. This is the first demonstration that sun leaves benefit more
from amphistomy because it compensates for developmental plasticity in
leaf structure.

Two outstanding areas for further research are 1) the specific
anatomical changes, such as leaf thickness or mesophyll porosity, that
increase the diffusion path length for CO\(_2\) in sun leaves, and 2)
the costs of amphistomy that could explain plasticity in \gmaxratio{} in
accessions from low-light habitats. Resolving the first question will
require detailed anatomical measurements on a large number of shade and
sun leaves with concomitant estimates of \aax. The fact that accessions
from low-light habitats are more plastic suggests that costs of
amphistomy are greatest in these environments. (not finished yet)

Problems to bring up in discussion: - all our species were amphi, so
they may all live in PAI below threshold to favor amphi - other
microclimate diffs not considered - don't know specific anatomical
change

\begin{enumerate}
\def\labelenumi{\arabic{enumi}.}
\setcounter{enumi}{7}
\tightlist
\item
  What it all means
\end{enumerate}

\begin{itemize}
\tightlist
\item
  Our study reveals an important role for plasticity in explaining why
  amphi leaves are more common in sunny habitats
\item
  Confirms that leaf structure of sun leaves incurs costs that
  amphistomy can mitigate
\item
  It will be useful to consider coordination between leaf anatomy,
  stomatal density, and ratio in future studies of leaf
  structure-function and in crops (or something)
\end{itemize}

\section{Materials and Methods
{[}\#sec-methods{]}}\label{materials-and-methods-sec-methods}

{[}this will be moved to SI eventually. note that much of it also the
same in solanum-aq project, so might change to cite that paper?{]}

(not sure where this goes) * We acclimated the focal leaf to high light
(\ppfdequals{2000}) and high relative humidity (\rhequals{70}) until
\(A\) and \gsw{} reach their maximum. After that, we decreased \rh{} to
\(\approx 10\%\) to induce rapid stomatal closure without biochemical
down

\subsection{Accessions}\label{accessions}

We compared AA among 29 ecologically diverse accessions of wild tomato,
including representatives of all described species of \emph{Solanum}
sect. \emph{Lycopersicon} and sect. \emph{Lycopersicoides} (\emph{51})
and the cultivated tomato \emph{S. lycopersicum} var.
\emph{lycopersicum} \autoref{tab:accessions}. Due to constraints on
growth space and time, we spread out measurements over 61.1 weeks from
August 29, 2022 to October 31, 2023. Replicates within accession were
evenly spread out over this period to prevent confounding of temporal
variation in growth conditions with accession. {[}anything else to say
here? maybe explain accession selection and phylogeny?{]}

\begin{longtable}{>{\raggedright\arraybackslash}p{3.25cm}lrrrr}

\caption{\label{tbl-accessions}Accessions of \emph{Solanum} used in this
study. The species name, accession number, collection latitude,
longitude, elevation, and daily solar radiation estimated using the
SPLASH algorithm (see `Climate data'). TGRC: Tomato Genetics Resource
Center; PPFD: Photosynthetic Photon Flux Density.}

\tabularnewline

\toprule
Species & TGRC accession & Latitude & Longitude & Elevation (mas) & $\mathrm{PAI}~(\unit{\meter\squared\per\meter\squared}$\\
\midrule
\em{\cellcolor{gray!10}{S. arcanum}} & \cellcolor{gray!10}{LA2172} & \cellcolor{gray!10}{-6.008} & \cellcolor{gray!10}{-78.858} & \cellcolor{gray!10}{662} & \cellcolor{gray!10}{0.7}\\
\em{S. cheesmaniae} & LA0429 & -0.644 & -90.329 & 800 & 0.6\\
\em{\cellcolor{gray!10}{S. cheesmaniae}} & \cellcolor{gray!10}{LA3124} & \cellcolor{gray!10}{-0.804} & \cellcolor{gray!10}{-90.042} & \cellcolor{gray!10}{1} & \cellcolor{gray!10}{0.4}\\
\em{S. chilense} & LA1782 & -15.267 & -74.633 & 1000 & 0.3\\
\em{\cellcolor{gray!10}{S. chilense}} & \cellcolor{gray!10}{LA4117A} & \cellcolor{gray!10}{-22.907} & \cellcolor{gray!10}{-67.941} & \cellcolor{gray!10}{3540} & \cellcolor{gray!10}{0.2}\\
\addlinespace
\em{S. chmielewskii} & LA1028 & -13.883 & -73.017 & 3000 & 0.4\\
\em{\cellcolor{gray!10}{S. chmielewskii}} & \cellcolor{gray!10}{LA1316} & \cellcolor{gray!10}{-13.400} & \cellcolor{gray!10}{-73.906} & \cellcolor{gray!10}{2920} & \cellcolor{gray!10}{1.1}\\
\em{S. corneliomulleri} & LA0107 & -13.117 & -76.383 & 60 & 0.0\\
\em{\cellcolor{gray!10}{S. corneliomulleri}} & \cellcolor{gray!10}{LA0444} & \cellcolor{gray!10}{-13.433} & \cellcolor{gray!10}{-76.133} & \cellcolor{gray!10}{100} & \cellcolor{gray!10}{0.5}\\
\em{S. galapagense} & LA0436 & -0.953 & -90.978 & 40 & 0.2\\
\addlinespace
\em{\cellcolor{gray!10}{S. galapagense}} & \cellcolor{gray!10}{LA1044} & \cellcolor{gray!10}{-0.284} & \cellcolor{gray!10}{-90.548} & \cellcolor{gray!10}{0} & \cellcolor{gray!10}{0.2}\\
\em{S. habrochaites} & LA0407 & -2.181 & -79.884 & 70 & 0.5\\
\em{\cellcolor{gray!10}{S. habrochaites}} & \cellcolor{gray!10}{LA1777} & \cellcolor{gray!10}{-9.550} & \cellcolor{gray!10}{-77.700} & \cellcolor{gray!10}{3216} & \cellcolor{gray!10}{0.4}\\
\em{S. huaylasense} & LA1358 & -9.533 & -77.967 & 750 & 0.6\\
\em{\cellcolor{gray!10}{S. huaylasense}} & \cellcolor{gray!10}{LA1360} & \cellcolor{gray!10}{-9.546} & \cellcolor{gray!10}{-77.929} & \cellcolor{gray!10}{1490} & \cellcolor{gray!10}{0.5}\\
\addlinespace
\em{S. huaylasense} & LA1364 & -10.133 & -77.383 & 2920 & 0.9\\
\em{\cellcolor{gray!10}{S. lycopersicoides}} & \cellcolor{gray!10}{LA2951} & \cellcolor{gray!10}{-19.317} & \cellcolor{gray!10}{-69.450} & \cellcolor{gray!10}{2200} & \cellcolor{gray!10}{0.5}\\
\em{S. lycopersicoides} & LA4126 & -19.287 & -69.396 & 3120 & 0.4\\
\em{\cellcolor{gray!10}{S. neorickii}} & \cellcolor{gray!10}{LA1322} & \cellcolor{gray!10}{-13.483} & \cellcolor{gray!10}{-72.442} & \cellcolor{gray!10}{2380} & \cellcolor{gray!10}{0.7}\\
\em{S. neorickii} & LA2133 & -3.400 & -79.183 & 1980 & 1.1\\
\addlinespace
\em{\cellcolor{gray!10}{S. pennellii}} & \cellcolor{gray!10}{LA0716} & \cellcolor{gray!10}{-16.225} & \cellcolor{gray!10}{-73.617} & \cellcolor{gray!10}{50} & \cellcolor{gray!10}{0.2}\\
\em{S. pennellii} & LA0750 & -14.775 & -75.034 & 550 & 0.1\\
\em{\cellcolor{gray!10}{S. pennellii}} & \cellcolor{gray!10}{LA3778} & \cellcolor{gray!10}{-14.775} & \cellcolor{gray!10}{-75.034} & \cellcolor{gray!10}{616} & \cellcolor{gray!10}{0.1}\\
\em{S. peruvianum} & LA2744 & -18.550 & -70.150 & 400 & 0.2\\
\em{\cellcolor{gray!10}{S. peruvianum}} & \cellcolor{gray!10}{LA2964} & \cellcolor{gray!10}{-18.028} & \cellcolor{gray!10}{-70.835} & \cellcolor{gray!10}{75} & \cellcolor{gray!10}{1.2}\\
\addlinespace
\em{S. pimpinellifolium} & LA1269 & -11.483 & -77.075 & 400 & 0.5\\
\em{\cellcolor{gray!10}{S. pimpinellifolium}} & \cellcolor{gray!10}{LA1589} & \cellcolor{gray!10}{-8.433} & \cellcolor{gray!10}{-78.817} & \cellcolor{gray!10}{30} & \cellcolor{gray!10}{0.1}\\
\em{S. pimpinellifolium} & LA2933 & -1.442 & -80.562 & 375 & 1.6\\
\em{\cellcolor{gray!10}{S. sitiens}} & \cellcolor{gray!10}{LA4116} & \cellcolor{gray!10}{-22.159} & \cellcolor{gray!10}{-68.782} & \cellcolor{gray!10}{2960} & \cellcolor{gray!10}{0.1}\\
\bottomrule

\end{longtable}

\subsection{Plant growth conditions}\label{plant-growth-conditions}

In all growth spaces, we recorded \(\mathrm{PPFD}\) using full spectrum
quantum sensors (SQ-500-SS, Apogee Instruments, Logan, Utah, USA); we
recorded temperature, RH, and {[}CO\(_2\){]} using an EE894 sensor (E+E
Elektronik, Engerwitzdorf, Austria) protected by a radiation shield. All
environmental measurements were taken every 10 minutes from the middle
of plants racks at approximately the same height as the leaves we
measured. We measured leaf temperature of focal leaves prior to
measurement using an infrared radiometer (SI-111-SS, Apogee Instruments,
Logan, Utah, USA).

\subsubsection{Germination and seedling
stage}\label{germination-and-seedling-stage}

Seeds provided by the Tomato Genetics Resource Center germinated on
moist paper in plastic boxes after soaking for 30-60 minutes in a 50\%
(volume per volume) solution of household bleach and water, followed by
a thorough rinse. We transferred seedlings to cell-pack flats containing
Pro-Mix BX potting mix (Premier Tech, Rivière-du-Loup, Quebec, Canada)
once cotyledons fully emerged, typically within 1-2 weeks of sowing. We
grew seeds and seedlings for both sun and shade treatments under the
same environmental conditions (12:12 h,
24.3:\(\qty{21.7}{\degreeCelsius}\), 49.6:58.4 RH day:night cycle). LED
light provided
\(\mathrm{PPFD} = \qty{267}{\micro\mol\raiseto{-2}\meter\raiseto{-1}\second}\)
(Fluence RAZRx, Austin, Texas, USA).

\subsubsection{Light treatments}\label{light-treatments}

Seedlings were randomly assigned in alternating order within accession
to the sun or shade treatment during transplanting. After seedlings
established in cell-pack flats for \(\approx 2\) weeks, we transplanted
them to 3.78 L plastic pots containing 60\% Pro-Mix BX potting mix, 20\%
coral sand (Pro-Pak, Honolulu, Hawaiʻi, USA), and 20\% cinders (Niu
Nursery, Honolulu, Hawaiʻi, USA). Percentage composition is on a volume
basis. The soil mixture contained slow release NPK fertilizer following
manufacturer instructions (Osmocote Smart-Release Plant Food Flower \&
Vegetable, The Scotts Company, Marysville, Ohio, USA). We determined pot
field capacity one week after transplanting using a scale (Ohaus V12P15
Valor 1000, Parsippany, New Jersey, USA) and watered to field capacity
three times per week to prevent drought stress.

We assigned sun and shade treatment to lower and upper racks of a
\(\qty{1.22}{\meter} \times \qty{2.44}{\meter}\) shelving unit in a
climate-controlled growth room. We assigned the sun treatment to the
lower rack to limit diffuse light from reaching the shade treatment. The
average daytime \ppfd{} was \ppfdqty{761} and \ppfdqty{115} for sun and
shade treatments, respectively. To isolate the effect of light intensity
from quality, we used the same LED model with the the same spectrum
(Fluence SPYDR 2i, Austin, Texas, USAS), but dimmed the lights in the
shade treatment. To maintain homogeneous environmental conditions other
than light, we mixed air within the growth room using an air circulator
(Vornado 693DC, Andover, Kansas, USA) and within racks using a miniature
oscillating air circulator (Vornado Atom 1, Andover, Kansas, USA).
Despite these efforts, the air in the sun treatment was on average
\(\qty{2.56}{\degreeCelsius}\) warmer and the average RH was
consequently 5.75 lower. However, because of evaporative cooling, the
leaves in the sun treatment were only \(\qty{0.886}{\degreeCelsius}\) on
average (\(n = 699\) leaves).

\subsection{Leaf trait measurements}\label{leaf-trait-measurements}

We selected a fully expanded, unshaded leaf at least six leaves above
the cotyledons during early vegetative growth. This typically meant that
plants had grown in light treatments for \(\approx\) 4 weeks, ensuring
they had time to sense and respond developmentally to the light
intensity of the treatment rather than the seedling conditions
(\emph{53}). Shade plants grew slower than sun plants, hence leaves at
the same developmental stage were measured on chronologically older
plants in the shade treatment. In some sun plants, we had to use leaves
higher on the stem because short internodes made lower leaves
inaccessible with the gas exchange equipment. We measured terminal
leaflets in 82.6\% of cases, but used the lateral leaflet closest to the
terminal leaflet when it was damaged or difficult to clamp into the gas
exchange chamber. When a leaflet was damaged during gas exchange
measurements, we collected anatomical data from the nearest leaflet on
the same leaf (1.58)\%. We measured chlorophyll concentration index
(CCI) using a chrolophyll concentration meter (MC-100, Apogee
Instruments, Logan, Utah, USA) on the lamina of focal leaflets before
gas exchange measurements at the same time we measured leaf temperature.

\subsubsection{Amphistomy advantage}\label{amphistomy-advantage}

We estimated `amphistomy advantage' (\aax) \emph{sensu} (\emph{20}), but
with modifications previously described in (\emph{46}). \aax{} is
calculated as the log-response ratio of \(A\) compared at the same total
\gsw:

\[\mathrm{AA} = \mathrm{log}(A_{\mathrm{amphi}} / A_{\mathrm{hypo}})\]

We measured the photosynthetic rate of an untreated amphistomatous leaf
(\Aamphi) over a range of \gsw{} values. We refer to this as an
\agcurve{} curve. We compared the \agcurve{} curve of the untreated leaf
to the photosynthetic rate of pseudohypostomatous leaf (\Ahypo), which
is the same leaf but with gas exchange through the upper surface blocked
by a neutral density plastic (propafilm).

We measured \agcurve{} curves using a portable infrared gas analyzer
(LI-6800PF, LI-COR Biosciences, Lincoln, Nebraska, USA).
Light-acclimated plants were placed under LEDs dimmed to match their
light treatment during gas exchange measurements. We estimated the
photosynthetic rate (\(A\)) and stomatal conductance to CO\(_2\) (\gsw)
at ambient CO\(_2\) (\caequals{415}) and \tleafequals{25.0}. The
irradiance of the light source in the pseudohypo leaf was higher because
the propafilm reduces transmission. To compensate for reduced
transmission, we increased incident \ppfd for pseudohypo leaves by a
factor 1/0.91, the inverse of the measured transmissivity of the
propafilm. We also set the stomatal conductance ratio, for purposes of
calculating boundary layer conductance, to 0 for pseudohypo leaves
following manufacturer directions.

We collected four \agcurve{} curves per leaf, an amphi (untreated) curve
and a pseudohypo (treated) curve at high light-intensity
(\ppfdequals{2000}; 97.8:2.24 red:blue) and low light-intensity
(\ppfdequals{150}; 87.0:13.0 red:blue). We always measured high
light-intensity curves first because photosynthetic downregulation is
faster than upregulation in these species. To control for order effects,
we alternated between starting with amphi or pseudohypo leaf
measurements. Unlike (\emph{46}), preliminary experiments with
\emph{Solanum} indicated a strong order effect in that \(A\) declined in
the second curve. Therefore, we made measurements over two days. On the
first day, we measured high and low light-intensity curves for either
amphi or pseudohypo leaves; on the second day, we measured high and low
light-intensity curves on the other leaf type.

In all cases, we acclimated the focal leaf to high light
(\ppfdequals{2000}) and high relative humidity (\rhequals{70}) until
\(A\) and \gsw{} reach their maximum. After that, we decreased \rh{} to
\(\approx 10\%\) to induce rapid stomatal closure without biochemical
downregulation. Hence, \Aamphi{} and \Ahypo{} were both measured at low
chamber humidity after the leaf had acclimated to high humidity. All
other environmental conditions in the leaf chamber remained the same. We
logged data until \gsw{} reached its nadir. We then acclimated the leaf
to low light (\ppfdequals{150}) and \rhequals{70} before inducing
stomatal closure with low \rh and logging data as described above.

\subsubsection{\texorpdfstring{Light-response (\aqcurve)
curves}{Light-response () curves}}\label{light-response-curves}

In 91.3\% of plants, we measured light-response (\aqcurve) curves on the
same leaflets as \agcurve{} curves. However, when a leaflet was damaged
during \agcurve{} curves, we used the next closest leaflet for
\aqcurve{} curves. Leaves acclimated to high light-intensity
(\ppfdequals{2000}), ambient CO\(_2\) (\caequals{415}), \rhequals{50},
and \tleafequals{25}. After \(A\) and \gsw{} stabilized, we measured
\(A\) at 20 light-intensity levels between \(0\) and
\(\qty{2000}{\micro\mol\raiseto{-2}\meter\raiseto{-1}\second}\) in
descending order.

\subsubsection{Stomatal anatomy}\label{stomatal-anatomy}

We estimated the stomatal density and size on ab- and adaxial leaf
surfaces from all leaves, using guard cell length as a proxy for
stomatal size since it proportional to maximum conductance (\emph{54}).
We made surface impressions of leaf lamina from the same area used for
gas exchange measurements using a-silicone impression material (Zhermack
elite HD+, light body, fast set, Rovigo, Italy). We applied clear nail
polish to make positive replicas of the impression. After nail polish
dried, we mounted replicas on a microscope slide using transparent tape
(\emph{55}). We digitized a portion of each leaf surface replica using a
brightfield microscope (Leica DM2000, Wetzlar, Germany). We counted and
measured guard cell length on all stomata using the FIJI implementation
of ImageJ2 version 2.3.0 (\emph{56}), then divided the count by the
visible leaf area (\(\qty{0.890}{\raiseto{2}\milli\meter}\)) to estimate
stomatal density.

\subsubsection{Leaf mass per area}\label{leaf-mass-per-area}

Leaf mass per area (LMA) is the dry mass divided by the leaflet area. We
scanned fresh leaflets on a flat bed scanner (Epson V600, Los Alamitos,
California. USA) and measured leaflet area from digital images using the
FIJI implementation of ImageJ2 version 2.3.0 (\emph{56}). We dried
leaves for 72 hours at \(\qty{74}{\degreeCelsius}\) in a food dehydrator
(Cosori CP267-FD, Vesync Co., Anaheim, California, USA) and weighed
using a benchtop analytical balance (Ohaus PR64 Analytical Balance,
Parsippany, New Jersey, USA). In \(10.5\%\) we measured LMA on the
adjacent leaflet because the focal leaflet was damaged or wilted while
making surface impressions and we could not reliably estimate area. LMA
data are missing from \(3.32\%\) of individuals because the area or mass
was not recorded at all or recorded incorrectly.

\subsection{\texorpdfstring{Cleaning \agcurve{}
curves}{Cleaning  curves}}\label{cleaning-curves}

The raw data set consisted of 2,370 \agcurve{} curves with an average of
63.2 points per curve. Manual curation of a data set this size in a
principled, consistent manner is not feasible. Therefore, we automated
data cleaning using custom \emph{R} scripts. Cleaning is divided into
six sequential steps (\autoref{tbl-cleaning1}).

\begin{table}

\caption{\label{tbl-cleaning1}Six sequential steps for cleaning
\agcurve{} curves. The rationale and procedure for each step are
described in the text. The rightmost columns summarize the number of
curves and mean number of points per curve remaining after each step.
For reference, there are four possible \agcurve{} curves per replicate:
all combinations of leaf type (amphi or pseudohypo) and light intensity
(high or low).}

\centering{

\centering
\begin{tabular}{>{\raggedright\arraybackslash}p{9cm}>{\centering\arraybackslash}p{2.5cm}>{\centering\arraybackslash}p{2.5cm}}
\toprule
Step: description & Number of curves & Number of points per curve\\
\midrule
\cellcolor{gray!10}{1. remove unreliable and unusable data points} & \cellcolor{gray!10}{2,361} & \cellcolor{gray!10}{63.0}\\
2. remove hysteretic portion of \agcurve{} curves at low \gsw{} & 2,360 & 59.2\\
\cellcolor{gray!10}{3. remove outliers within each \agcurve{} curve} & \cellcolor{gray!10}{2,360} & \cellcolor{gray!10}{58.7}\\
4. remove replicates with no overlap between amphi and pseudohypo \agcurve{} curves & 2,268 & 58.5\\
\cellcolor{gray!10}{5. thin redundant data points within each \agcurve{} curve} & \cellcolor{gray!10}{2,268} & \cellcolor{gray!10}{28.1}\\
\addlinespace
6. trim extreme \aax{} values & 2,214 & 28.1\\
\bottomrule
\end{tabular}

}

\end{table}%

\subsubsection{Remove unreliable and unusable data
points}\label{remove-unreliable-and-unusable-data-points}

\emph{Rationale}: Unreliable data points consisted of those where
chamber {[}CO\(_2\){]} was unstable and therefore measurements are not
biologically meaningful. Unusable data points were those where \(A < 0\)
because the logarithm of a negative number is undefined.

\emph{Procedure}: We retained data points where \cabetween{410}{420} and
\(A> 0\).

\subsubsection{\texorpdfstring{Remove hysteretic portion of \agcurve{}
curves at low
\gsw{}}{Remove hysteretic portion of  curves at low }}\label{remove-hysteretic-portion-of-curves-at-low}

\emph{Rationale}: In most \agcurve{} curves, we observed a hysteretic
response at low \gsw. After \gsw{} and \(A\) declined simultaneously,
\(A\) increased slightly as \gsw{} continued to decline or stabilize,
indicating some leaf acclimation to low \rh. We removed this portion of
the curve to focus curve-fitting on the primary domain where \(A\)
increases monotonically with \gsw{}.

\emph{Procedure}: For each curve, we removed data points after \gsw{}
had reached its minimum unless there were fewer than 10 data points
remaining.

\subsubsection{\texorpdfstring{Remove outliers within each \agcurve{}
curve}{Remove outliers within each  curve}}\label{remove-outliers-within-each-curve}

\emph{Rationale}: Individual outliers within \agcurve{} curves, usually
caused by transitory changes in chamber conditions, exert undue leverage
on parameter estimates and cause bias and/or low precision in parameter
estimates.

\emph{Procedure}: We fit provisional quadratic regressions for each
curve using ordinary least squares with the \texttt{lm()} function in
\emph{R}. We sequentially removed data points with an absolute external
studentized residual \(> 3\) until none remained.

\subsubsection{\texorpdfstring{Thin redundant data points within each
\agcurve{}
curve}{Thin redundant data points within each  curve}}\label{thin-redundant-data-points-within-each-curve}

\emph{Rationale}: Data points closely spaced along the \agcurve{} curve
provide redundant information and may be highly correlated
(i.e.~pseudoreplication). This occurred because data was logged at a
constant temporal interval, but the rate at which \gsw{} declined was
not constant. Thinning reduces parameter estimation bias toward densely
sampled regions of the curve which may not be the most biologically
informative.

\emph{Procedure}: We retained the maxima and minima \gsw{} for each
curve and thinned all but one point per thinning interval of
\(0.05~\log(\si{\mol\raiseto{-2}\meter\raiseto{-1}\second})\), retaining
the point nearest the midpoint of the interval.

\subsubsection{\texorpdfstring{Remove replicates with no overlap between
amphi and pseudohypo \agcurve{}
curves}{Remove replicates with no overlap between amphi and pseudohypo  curves}}\label{remove-replicates-with-no-overlap-between-amphi-and-pseudohypo-curves}

\emph{Rationale}: We could not estimate \aax{} for replicates where
amphi and pseudohypo \agcurve{} curves did not overlap.

\emph{Procedure}: We removed replicates where the range of \gsw{} values
for amphi and pseudohypo \agcurve{} curves did not overlap.

\subsubsection{\texorpdfstring{Trim extreme \aax{}
values}{Trim extreme  values}}\label{trim-extreme-values}

\emph{Rationale}: Extreme \aax{} values were likely due to measurement
error or leaf damage. Since amphi and pseudohypo \agcurve{} curves are
measured on consecutive days, a poor calibration or a damaged leaf could
cause a large difference in \(A\) between days, which would appear as an
extreme \aax{} value.

\emph{Procedure}: We provionsally estimated \aax{} for each replicate by
integrating over the range of \gsw{} values where amphi and pseudohypo
\agcurve{} curves overlap. In this procedure, curve parameters were
provisionally estimated using ordinary least squares with the
\texttt{lm()} function in \emph{R}. We then used point estimates of
\aax{} for each replicate as the response variable in a linear model
with light treatment, light intensity, accession, and all interactions
as explanatory variables. This model was also fit using ordinary least
squares with the \texttt{lm()} function in \emph{R}. We classified
extreme \aax{} values as those with an absolute internal studentized
residual \(> 3\). Because these values likely indicate significant
measurement error or leaf damage, we removed \agcurve{} curves at both
light intensities if either was classified as extreme.

\subsection{\texorpdfstring{Cleaning \aqcurve{}
curves}{Cleaning  curves}}\label{cleaning-curves-1}

The raw data set consisted of 658 \aqcurve{} curves with an average of
19.4. Manual curation of a data set this size in a principled,
consistent manner is not feasible. Therefore, we automated data cleaning
using custom \emph{R} scripts. Cleaning is divided into two sequential
steps (\autoref{tbl-cleaning2}).

\begin{table}

\caption{\label{tbl-cleaning2}Two sequential steps for cleaning
\agcurve{} curves. The rationale and procedure for each step are
described in the text. The rightmost columns summarize the number of
curves and mean number of points per curve remaining after each step.}

\centering{

\centering
\begin{tabular}{>{\raggedright\arraybackslash}p{9cm}>{\centering\arraybackslash}p{2.5cm}>{\centering\arraybackslash}p{2.5cm}}
\toprule
Step: description & Number of curves & Number of points per curve\\
\midrule
\cellcolor{gray!10}{1. remove outliers within each \aqcurve{} curve} & \cellcolor{gray!10}{658} & \cellcolor{gray!10}{19.1}\\
2. remove \aqcurve{} curves with poor fit & 652 & 19.1\\
\bottomrule
\end{tabular}

}

\end{table}%

\subsubsection{\texorpdfstring{Remove outliers within each \aqcurve{}
curve}{Remove outliers within each  curve}}\label{remove-outliers-within-each-curve-1}

\emph{Rationale}: Individual outliers within \agcurve{} curves, usually
caused by transitory changes in chamber conditions, exert undue leverage
on parameter estimates and cause bias and/or low precision in parameter
estimates.

\emph{Procedure}: We fit provisional nonrectangular hyperbola
(\emph{57}) to each \aqcurve{} curve using nonlinear regression with the
\texttt{nlsLM()} function from the \emph{R} package \textbf{minpack.lm}
version 1.2.4 (\emph{58}). We sequentially removed data points with an
absolute external studentized residual \(> 3\) until none remained.

\subsubsection{\texorpdfstring{Remove \aqcurve{} curves with poor
fit}{Remove  curves with poor fit}}\label{remove-curves-with-poor-fit}

\emph{Rationale}: \aqcurve{} curves with a poor fit to the
nonrectangular hyperbola most likely indicate systematic measurement
error and/or the leaf was not fully acclimated to the chamber
environment.

\emph{Procedure}: As described above, we fit provisional nonrectangular
hyperbola to each \aqcurve{} curve and calculated the model \(r^2\).
There was a clear break between typical curves and poorly fitting curves
where \(r^2 < 0.99\). We therefore removed \aqcurve{} curves with
\(r^2 < 0.99\).

\subsection{\texorpdfstring{Bayesian data analysis in
\emph{Stan}}{Bayesian data analysis in Stan}}\label{bayesian-data-analysis-in-stan}

We fit five models to test predictions of competing hypotheses about why
amphistomy advantage (\aax) might be greater for leaves in sunny, open
habitats. This section provides an overview of differences among models
(\autoref{tbl-models}). The next sections describe how we fit models in
\emph{Stan}, all model parameters and priors, and specific predictions
about parameter values for each hypothesis.

\begin{table}

\caption{\label{tbl-models}Summary of differences among competing models
of how \aax{} varies with light intensity, light treatment, and among
accessions as a function of native \ppfd. The models are numbered from
simpler to more complex. All models include fixed effects of light
intensity and light treatment; some models include interactions between.
All models include a phylogenetic random effect of accession on \aax;
some models include varying effects of light intensity and light
treatment among accessions. The last column indicates the
accession-level \aax{} variable we used as a response to native \ppfd.}

\centering{

\begin{tabular}{>{\centering\arraybackslash}p{0.3in}>{\raggedright\arraybackslash}p{1.3in}>{\raggedright\arraybackslash}p{3.4in}>{\raggedright\arraybackslash}p{1in}}
\toprule
Model & Fixed effects & Phylogenetic random effects & Response to native \ppfd\\
\midrule
1 & light intensity & varying intercept among accessions & $\mathrm{AA}_{0, \text{acc}}$\\

 & light treatment &  & \\
\cmidrule{1-4}
2 & light intensity & varying intercept among accessions & $\mathrm{AA}_{0, \text{acc}}$\\

 & light treatment &  & \\

 & intensity $\times$ treatment &  & \\
\cmidrule{1-4}
3 & light intensity & varying intercept among accessions & $\mathrm{AA}_{2000, \text{acc}}$\\

 & light treatment & varying effect of high light intensity among accessions & \\
\cmidrule{1-4}
4 & light intensity & varying intercept among accessions & $\mathrm{AA}_{\text{sun, acc}}$\\

 & light treatment & varying effect of sun treatment among accessions & \\
\cmidrule{1-4}
5 & light intensity & varying intercept among accessions & $\mathrm{AA}_{2000, \text{sun, acc}}$\\

 & light treatment & varying effect of high light intensity among accessions & \\

 & intensity $\times$ treatment & varying effect of sun treatment among accessions & \\
\bottomrule
\end{tabular}

}

\end{table}%

\subsubsection{\texorpdfstring{Fitting models in
\emph{Stan}}{Fitting models in Stan}}\label{sec-fitting}

We fit Bayesian models with MCMC sampling in the probabilistic
programming language \emph{Stan} (\emph{59}) using the \emph{R} package
\textbf{brms} version 2.22.0 (\emph{60}). We used CmdStan version 2.36.0
and \textbf{cmdstanr} version 0.9.0 (\emph{61}) to interface with
\emph{R} version 4.5.0 (\emph{62}). We sampled the posterior
distribution from 1 chains with 1000 iterations each after 1000 warmup
iterations per chain. We estimated parameters and confidence intervals
as the median and 95\% quantile intervals of the posterior,
respectively. We chose the number of chains, warmup and sampling
iterations, and maximum treedepth so that parameter estimates converged
(\(\hat{R} < 1.01\) (\emph{63})) and the effective sample size (ESS) for
each parameter was \(> 10^3\).

\subsubsection{Parameter estimation and priors}\label{sec-parameters}

There were four levels of parameter estimation in our analysis:

\begin{enumerate}
\def\labelenumi{\arabic{enumi}.}
\tightlist
\item
  Estimate \agcurve{} curve parameters
\item
  Estimate \aax{} for each light intensity with leaf using \agcurve{}
  curve parameters
\item
  Estimate the effects of light intensity, light treatment, and
  accession on \aax{} (assimilatory and plasticity hypotheses)
\item
  Estimate the effects of native light habitat on accession-level \aax{}
  (constitutive hypothesis)
\end{enumerate}

Although the higher-level parameter estimates depend on the lower-level
parameter estimates, we fit all models simultaneously to ensure that the
uncertainty in lower-level estimates propagated to higher-level
estimates.

\begin{longtable}{>{\raggedright\arraybackslash}p{1in}>{\raggedright\arraybackslash}p{5in}}

\caption{\label{tbl-parameters}Description of parameters estimated in
the hierarchical Bayesian model. The \texttt{Parameter} column lists the
parameter name as it appears in text. The \texttt{Description} column
provides a brief description of the parameter.}

\tabularnewline

\toprule
Parameter & Description\\
\midrule
\addlinespace[0.3em]
\multicolumn{2}{l}{\textbf{\agcurve{} curve parameters}}\\
\hspace{1em}\cellcolor{gray!10}{$\mathbf{B}_\text{curve}$} & \cellcolor{gray!10}{$n_\text{curve} \times 3$ array of random \agcurve{} curve-level coefficients ($b_{0,j}$, $b_{1,j}$, $b_{2,j}$); $\mathbf{B}_\text{curve} \sim \text{MVN}(\vec{0}, \symbf{\Sigma}_\text{curve})$}\\
\hspace{1em}$\vec{\beta}_\text{curve}$ & vector of mean quadratic coefficients ($\beta_0$, $\beta_1$, $\beta_2$)\\
\hspace{1em}\cellcolor{gray!10}{$\symbf{\Sigma}_\text{curve}$} & \cellcolor{gray!10}{$3 \times 3$ covariance matrix of curve-level coefficients}\\
\hspace{1em}$\sigma_{\qty{6}{\raiseto{2}\centi\meter},\epsilon}$ & minimum residual standard deviation when the measured leaf surface area is $\qty{6}{\raiseto{2}\centi\meter}$\\
\hspace{1em}\cellcolor{gray!10}{$\beta_{S,\epsilon}$} & \cellcolor{gray!10}{slope of the relationship between residual standard deviation and measured leaf surface area (log-link scale)}\\
\hspace{1em}$\rho_{\epsilon}$ & lag-1 residual autocorrelation\\
\addlinespace[0.3em]
\multicolumn{2}{l}{\textbf{\aax{} for each light intensity with leaf using agcurve{} curve parameters}}\\
\hspace{1em}\cellcolor{gray!10}{$\widehat{\mathrm{AA}}_{klm}$} & \cellcolor{gray!10}{estimate of \aax{} for the $k^{\text{th}}$ leaf at light intensity $l$ in accession $m$}\\
\addlinespace[0.3em]
\multicolumn{2}{l}{\textbf{effects of light intensity, light treatments, and accession on \aax}}\\
\hspace{1em}$\beta_{\mathrm{AA},0}$ & intercept of \aax{} at low light intensity in shade treatment\\
\hspace{1em}\cellcolor{gray!10}{$\beta_{\mathrm{AA},2000}$} & \cellcolor{gray!10}{effect of high light intensity at \ppfdequals{2000} on \aax}\\
\hspace{1em}$\beta_{\mathrm{AA},\text{sun}}$ & effect of sun treatment on \aax{}\\
\hspace{1em}\cellcolor{gray!10}{$\beta_{\mathrm{AA},2000,\text{sun}}$} & \cellcolor{gray!10}{effect of high light intensity at \ppfdequals{2000} on \aax{} in sun treatment}\\
\hspace{1em}$\vec{\beta}_{\mathrm{AA},\text{acc}}$ & vector of $n_\text{acc}$ phylogenetically structured random accession-level effects on \aax; $\vec{\beta}_{\mathrm{AA},\text{acc}} \sim \text{MVN}(\vec{0}, \symbf{\Sigma}_{\text{AA,acc}})$\\
\hspace{1em}\cellcolor{gray!10}{$\symbf{\Sigma}_{\text{AA,acc}}$} & \cellcolor{gray!10}{$n_\text{acc} \times n_\text{acc}$ covariance matrix of phylogenetically structured random accession-level effects on \aax;}\\
\hspace{1em}$\vec{\beta}_{\mathrm{AA},\text{rep}}$ & vector of $n_\text{rep}$ random replicate-level effects on \aax; $\vec{\beta}_{\mathrm{AA},\text{rep}} \sim \text{Normal}(0, \sigma_{\mathrm{AA},\text{rep}})$\\
\hspace{1em}\cellcolor{gray!10}{$\vec{\beta}_{\mathrm{AA},\text{2000,acc}}$} & \cellcolor{gray!10}{vector of $n_\text{acc}$ phylogenetically structured random accession-level effects of high light intensity at \ppfdequals{2000} on \aax; $\vec{\beta}_{\mathrm{AA},\text{2000,acc}} \sim \text{MVN}(\vec{0}, \symbf{\Sigma}_{\mathrm{AA},\text{2000,acc}})$}\\
\hspace{1em}$\vec{\beta}_{\mathrm{AA},\text{sun,acc}}$ & vector of $n_\text{acc}$ phylogenetically structured random accession-level effects of sun treatment on \aax; $\vec{\beta}_{\mathrm{AA},\text{sun,acc}} \sim \text{MVN}(\vec{0}, \symbf{\Sigma}_{\mathrm{AA},\text{sun,acc}})$\\
\hspace{1em}\cellcolor{gray!10}{$\sigma_{\mathrm{AA},\epsilon,0}$} & \cellcolor{gray!10}{intercept of phylogenetically unstructured residual standard deviation of \aax}\\
\hspace{1em}$\beta_{\mathrm{AA},\epsilon,2000}$ & effect of high light intensity at \ppfdequals{2000} on phylogenetically unstructured residual standard deviation of \aax{} (log-link scale)\\
\hspace{1em}\cellcolor{gray!10}{$\beta_{\mathrm{AA},\epsilon,\text{sun}}$} & \cellcolor{gray!10}{effect of sun treatment on phylogenetically unstructured residual standard deviation of \aax{} (log-link scale)}\\
\hspace{1em}$\sigma_{\mathrm{AA},\text{rep}}$ & standard deviation of random replicate-level effects on \aax\\
\hspace{1em}\cellcolor{gray!10}{$\alpha_{\mathrm{AA},\text{acc}}$} & \cellcolor{gray!10}{decay rate of phylogenetic covariance in random accession-level effects on \aax}\\
\hspace{1em}$\sigma^2_{\mathrm{AA},\text{acc}}$ & phylogenetic diffusion rate in random accession-level effects on \aax\\
\hspace{1em}\cellcolor{gray!10}{$\alpha_{\mathrm{AA},\text{2000,acc}}$} & \cellcolor{gray!10}{decay rate of phylogenetic covariance in random accession-level effects of high light intensity at \ppfdequals{2000} on \aax}\\
\hspace{1em}$\sigma^2_{\mathrm{AA},\text{2000,acc}}$ & phylogenetic diffusion rate in random accession-level effects of high light intensity at \ppfdequals{2000} on \aax\\
\hspace{1em}\cellcolor{gray!10}{$\alpha_{\mathrm{AA},\text{sun,acc}}$} & \cellcolor{gray!10}{decay rate of phylogenetic covariance in random accession-level effects of sun treatment on \aax}\\
\hspace{1em}$\sigma^2_{\mathrm{AA},\text{sun,acc}}$ & phylogenetic diffusion rate in random accession-level effects of sun treatment on \aax\\
\addlinespace[0.3em]
\multicolumn{2}{l}{\textbf{effects of native light habitat on accession-level \aax}}\\
\hspace{1em}\hspace{1em}\cellcolor{gray!10}{$\beta_{\mathrm{AA},\text{PPFD},0}$} & \cellcolor{gray!10}{intercept of accession-level \aax{} when native \ppfdequals{0}}\\
\hspace{1em}\hspace{1em}$\beta_{\mathrm{AA},\text{PPFD},1}$ & slope of native \ppfd{} on accession-level \aax{}\\
\hspace{1em}\cellcolor{gray!10}{$\alpha_{\mathrm{AA},\text{PPFD}}$} & \cellcolor{gray!10}{decay rate of phylogenetic covariance in residuals of model testing effect of native \ppfd{} on accession-level \aax}\\
\hspace{1em}$\sigma^2_{\mathrm{AA},\text{PPFD}}$ & phylogenetic diffusion rate in residuals of model testing effect of native \ppfd{} on accession-level \aax\\
\bottomrule

\end{longtable}

\paragraph{\texorpdfstring{\agcurve{} curve
parameters}{ curve parameters}}\label{curve-parameters}

We modeled \logA as a quadratic function of \loggsw for each leaf using
the following equation:

\[\log(A_{ij}) = (\beta_0 + b_{0,j}) + (\beta_1 + b_{1,j}) \log{g_{\text{sw},i}} + (\beta_2 + b_{2,j}) \log{g_{\text{sw},i}}^2 + \epsilon_{i}\]

where \(\beta_0\), \(\beta_1\), and \(\beta_2\) are the average
intercept, linear, and quadratic coefficients, respectively. We used
diffuse normal priors with mean \(0\) and standard deviation \(10\) on
these parameters. We estimated random effects of curve \(j\) on the
intercept (\(b_{0,j}\)), linear (\(b_{1,j}\)), and quadratic
(\(b_{2,j}\)) coefficients. We assumed that the \(j \times 3\) array of
coefficients was multivariate normal a mean vector of \(\vec{0}\) and
covariance \(\Sigma_\text{curve}\). We used a weakly informative normal
prior with mean \(0\) and standard deviation \(1\) on the
log-transformed standard deviations (i.e.~the diagonal of
\(\Sigma_\text{curve}\)). We used a weakly informative
\(\mathrm{LJK}(2)\) prior on the correlation matrix. The off diagonal
elements of \(\Sigma_\text{curve}\) can be calculated from its diagonal
elements and the correlation matrix.

The residuals \(\epsilon_{i}\) were modeled as a lag-1 autocorrelated
time-series. We further assumed that the residual standard deviation of
the \(j^{\text{th}}\) curve (\(\sigma_{\epsilon,j}\)) was inversely
proportional to the leaf surface area (\(S_j\)) within the chamber:

\[\log (\sigma_{\epsilon,j}) = \log (\sigma_{\qty{6}{\raiseto{2}\centi\meter},\epsilon}) + (6 - S_j) \beta_{S,\epsilon}\]
where \(\sigma_{\qty{6}{\raiseto{2}\centi\meter},\epsilon}\) is the
minimum residual standard deviation when the
\(\qty{6}{\raiseto{2}\centi\meter}\) chamber is completely filled. The
residual standard deviation increases on log-linear scale by
\(\beta_{S,\epsilon}\). We used a weakly informative normal prior with
mean \(-3\) and standard deviation \(5\) on
\(\log (\sigma_{\qty{6}{\raiseto{2}\centi\meter},\epsilon})\) and a
weakly informative normal prior with mean \(0\) and standard deviation
\(1\) on \(\beta_{S,\epsilon}\).

\paragraph{\texorpdfstring{\aax{} for each light intensity with leaf
using \agcurve{} curve
parameters}{ for each light intensity with leaf using  curve parameters}}\label{for-each-light-intensity-with-leaf-using-curve-parameters}

Within the \(k^{\text{th}}\) leaf, we estimated \aax{} for each light
intensity by integrating the difference in \logA{} between the amphi and
pseudohypo \agcurve{} curves over the range of \gsw{} values where the
curves overlap (from \(\min(\log(g_\text{sw}))\) to
\(\max(\log(g_\text{sw}))\)). The estimate of \aax{} for the
\(k^{\text{th}}\) leaf at light intensity \(l\) in accession \(m\) is:

\[\widehat{\mathrm{AA}}_{klm} = \int_{\text{min(log(}g_\text{sw}))}^{\text{max(log(}g_\text{sw}))} \text{log}\bigg(\frac{\hat{A}_\text{amphi}(x; \theta_{klm,\text{amphi})}}{\hat{A}_\text{hypo}(x; \theta_{klm,\text{hypo})}}\bigg) dx\]
where:

\[\theta_\text{amphi} \in \{\hat{b}_{0, f(\text{amphi}, k,l,m)}, \hat{b}_{1, f(\text{amphi}, k,l,m)}, \hat{b}_{2, f(\text{amphi}, k,l,m)}\}, \text{and}\]

\[\theta_\text{hypo} \in \{\hat{b}_{0, f(\text{hypo}, k,l,m)}, \hat{b}_{1, f(\text{hypo}, k,l,m)}, \hat{b}_{2, f(\text{hypo}, k,l,m)}\}.\]
The function \(f : \Theta_1 \rightarrow \Theta_2\) maps the set
\(\Theta_1\) indexed by leaf type (amphi or pseudohypo), leaf replicate,
light intensity, and accession to set \(\Theta_2\) indexed by individual
\agcurve{} curve. This mapping is necessary because the random effects
structure differs between models of \loggsw{} on \logA{} and that of
models predicting \aax{} described in the next section.

\paragraph{\texorpdfstring{Effects of light intensity, light treatment,
and accession on
\aax}{Effects of light intensity, light treatment, and accession on }}\label{effects-of-light-intensity-light-treatment-and-accession-on}

We tested for effects of light intensity, light treatment, accession,
and their interactions on \aax. All models included effects of light
intensity and light treatment, as well as random effects of accession
and replicate within accession. More complex models included
interactions between light intensity and light treatment, as well as
random effects of accession on the effects of light intensity and light
treatment. We used a weakly informative normal prior with mean \(0\) and
standard deviation \(10\) for fixed effects of light intensity and
treatment. We used a weakly informative normal prior with mean \(-3\)
and standard deviation \(5\) for the the random effect standard
deviation of replicate within accession. We accounted for the
phylogenetic structure among the random effects of accession using an
Ornstein-Uhlenbeck (OU) process (\emph{64}). The expected covariance
between accessions \(i\) and \(j\) (\(\text{Cov}(i, j)\) ) is:

\[\text{Cov}(i, j) = \frac{\sigma^2}{2 \alpha} \exp(-\alpha D_{ij})\]
where \(\sigma^2\) is the variance of the random effect, \(\alpha\) is
the rate of decay of the covariance with phylogenetic distance,
\(D_{ij}\), the phylogenetic distance between accessions \(i\) and
\(j\). We estimated \(\sigma^2 / (2 \alpha)\) and \(\alpha\) as a
separate parameters and reparameterized them as \(\sigma^2\) and
\(\alpha\). We used a weakly informative normal prior with mean \(0\)
and standard deviation \(10\) on OU parameters.

We modeled the residual standard deviation of \aax,
(i.e.~phylogenetically unstructured variation unaccounted for by
explanatory variables) on a log-link scale with effects of light
intensity and light treatment. We used a weakly informative normal prior
with mean \(-3\) and standard deviation \(5\) on the residual standard
deviation intercept and a weakly informative normal prior with mean
\(0\) and standard deviation \(1\) on the effects of light intensity and
light treatment on the residual standard deviation.

\paragraph{\texorpdfstring{Effects of native light habitat on
accession-level
\aax}{Effects of native light habitat on accession-level }}\label{effects-of-native-light-habitat-on-accession-level}

We tested a linear effect of native light habitat on accession-level
\aax. In Models 1 and 2, we used the random intercept of \aax{} as a
response variable; in model 3 we used the random intercept plus the
random effect of accession at high light intensity; in model 4 we used
the random intercept plus the random effect of accession in the sun
treatment; in model 5 we used the random intercept plus the random
effects of accession at high light intensity and sun treatment. We used
a weakly informative normal prior with mean \(0\) and standard deviation
\(1\) for the slope and intercept. We accounted for the phylogenetic
structure among the model residuals using an (OU) process. We used a
weakly informative normal prior with mean \(0\) and standard deviation
\(10\) on OU parameters.

\subsubsection{Predictions}\label{sec-predictions}

The assimilatory, plastic, and constitutive hypotheses make different
predictions about the relationship between \aax{} and light intensity,
light treatment, and native \ppfd{} among accessions. Since these
hypotheses are not mutually exclusive, we describe how we assessed
support for one, two, or all three hypotheses simultaneously in
\autoref{tbl-predictions}. In general, the assimilatory hypothesis was
supported if \aax{} was greater at high light intensity than low light
intensity. The plastic hypothesis was supported if \aax{} was greater in
sun leaves than shade leaves. The constitutive hypothesis was supported
if accession-level \aax{} increased with native \ppfd. In interactive
models, we only consider positively reinforcing interactions between
high light intensity, sun leaves, and native \ppfd{} because these are
the only interactions which could explain why amphistomatous leaves are
advantageous in high light habitats.

We evaluated predictions using a combination of parameter estimation and
model selection. In all cases, we estimated parameters and confidence
intervals as described in Section~\ref{sec-fitting}. If the 95\%
confidence intervals for a parameter did not overlap zero, we considered
the parameter to be significantly different from zero. We used the
leave-one-out cross-validation information criterion (LOOIC) to compare
the fit of models (\autoref{tbl-models}) using the \emph{R} package
\textbf{loo} version 2.8.0 (\emph{65}) to calculate LOOIC values. Note
that LOOIC was calculated only from pointwise likelihood values of the
submodel estimating effects of light intensity, light treatments, and
accession on \aax{} (see Section~\ref{sec-parameters}). We considered
models with two standard errors of the mean lower LOOIC value to be a
better fit to the data; we considered models with LOOIC values within
two standard errors of the mean to be have similar support.

\begin{longtable}{>{\raggedright\arraybackslash}p{1in}>{\raggedright\arraybackslash}p{1.5in}>{\raggedright\arraybackslash}p{3in}}

\caption{\label{tbl-predictions}Predictions of competing hypotheses
about the relationship between \aax{} and light intensity, light
treatment, and native \ppfd{} among accessions. The middle column lists
specific directional predictions about parameter values and model fit
according to leave-one-out cross-validation information criterion
(LOOIC), where \(\text{LOOIC}_i\) is the LOOIC value for model \(i\).
The rightmost column describes the predictions in words and explains how
accession-level \aax{} is calculated in the relevant model.}

\tabularnewline

\toprule
\textbf{Hypothesis} & \textbf{Prediction(s)} & \textbf{Description}\\
\midrule
Assimilatory & $\beta_{\mathrm{AA},2000} > 0$ & \hspace{-1em}Average \aax{} at high light intensity is greater than that at low light intensity\\
\cmidrule{1-3}\pagebreak[0]
Plastic & $\beta_{\mathrm{AA},\text{sun}} > 0$ & \hspace{-1em}Average \aax{} in sun leaves is greater than that in shade leaves\\
\cmidrule{1-3}\pagebreak[0]
Constitutive & $\beta_{\mathrm{PPFD,AA}} > 0$ & \hspace{-1em}Accession-level \aax{} ($\mathrm{AA}_\text{acc}$) increases with native \ppfd\\
\nopagebreak
 &  & \hspace{-1em}$\mathrm{AA}_\text{acc} = \beta_{\mathrm{AA}, 0} + \vec{\beta}_{\mathrm{AA}, \text{acc}}$\\
\cmidrule{1-3}\pagebreak[0]
Assimilatory $\times$ Plastic & $\beta_{\mathrm{AA},2000,\text{sun}} > 0$ & \hspace{-1em}Average \aax{} is highest at high light intensity in sun leaves\\
\nopagebreak
 & $\text{LOOIC}_\text{1} > \text{LOOIC}_\text{2}$ & \hspace{-1em}\\
\cmidrule{1-3}\pagebreak[0]
Assimilatory $\times$ Constitutive & $\beta_{\mathrm{AA},2000} > 0$ & \hspace{-1em}Average \aax{} at high light intensity is greater than that at low light intensity\\
\nopagebreak
 & $\beta_{\mathrm{PPFD,AA}} > 0$ & \hspace{-1em}Accession-level \aax{} at high light intensity ($\mathrm{AA}_{\text{acc},2000}$) increases with native \ppfd\\
\nopagebreak
 & $\text{LOOIC}_\text{1} > \text{LOOIC}_\text{3}$ & \hspace{-1em}$\mathrm{AA}_{\text{acc},2000} = \beta_{\mathrm{AA}, 0} + \vec{\beta}_{\mathrm{AA}, \text{acc}} + \vec{\beta}_{\mathrm{AA}, 2000, \text{acc}}$\\
\cmidrule{1-3}\pagebreak[0]
Plastic $\times$ Constitutive & $\beta_{\mathrm{AA},\text{sun}} > 0$ & \hspace{-1em}Average \aax{} in sun leaves is greater than that in shade leaves\\
\nopagebreak
 & $\beta_{\mathrm{PPFD,AA}} > 0$ & \hspace{-1em}Accession-level \aax{} in sun leaves ($\mathrm{AA}_{\text{acc},\text{sun}}$) increases with native \ppfd\\
\nopagebreak
 & $\text{LOOIC}_\text{1} > \text{LOOIC}_\text{4}$ & \hspace{-1em}$\mathrm{AA}_{\text{acc},\text{sun}} = \beta_{\mathrm{AA}, 0} + \vec{\beta}_{\mathrm{AA}, \text{acc}} + \vec{\beta}_{\mathrm{AA}, \text{sun}, \text{acc}}$\\
\cmidrule{1-3}\pagebreak[0]
Assimilatory $\times$ Plastic $\times$ Constitutive & $\beta_{\mathrm{AA},2000,\text{sun}} > 0$ & \hspace{-1em}Average \aax{} is highest at high light intensity in sun leaves\\
\nopagebreak
 & $\beta_{\mathrm{PPFD,AA}} > 0$ & \hspace{-1em}Accession-level \aax{} at high light intensity in sun leaves ($\mathrm{AA}_{\text{acc},2000,\text{sun}}$) increases with native \ppfd\\
\nopagebreak
 & $\text{LOOIC}_\text{3} > \text{LOOIC}_\text{5}$ & \hspace{-1em}$\mathrm{AA}_{\text{acc},2000,\text{sun}} = \beta_{\mathrm{AA}, 0} + \vec{\beta}_{\mathrm{AA}, \text{acc}} + \vec{\beta}_{\mathrm{AA}, 2000, \text{acc}} + \vec{\beta}_{\mathrm{AA}, \text{sun}, \text{acc}}$\\
\bottomrule

\end{longtable}

\section{Acknowledgements}\label{acknowledgements}

Justin Alter, Max Gatlin, Joana Kim, Jenna Matsuyama, Brandon Najarian,
Kai Yasuda. Sarah Friedrich helped with figures.

\section*{References}\label{references}
\addcontentsline{toc}{section}{References}

\phantomsection\label{refs}
\begin{CSLReferences}{0}{1}
\bibitem[\citeproctext]{ref-raven_selection_2002}
\CSLLeftMargin{1. }%
\CSLRightInline{J. A. Raven,
\href{https://doi.org/10.1046/j.0028-646X.2001.00334.x}{Selection
pressures on stomatal evolution}. \emph{New Phytologist} \textbf{153},
371--386 (2002).}

\bibitem[\citeproctext]{ref-mcadam_stomata_2021}
\CSLLeftMargin{2. }%
\CSLRightInline{S. A. M. McAdam, J. G. Duckett, F. C. Sussmilch, S.
Pressel, K. S. Renzaglia, R. Hedrich, T. J. Brodribb, A. Merced,
\href{https://doi.org/10.1002/ajb2.1619}{Stomata: The holey grail of
plant evolution}. \emph{American Journal of Botany} \textbf{108},
366--371 (2021).}

\bibitem[\citeproctext]{ref-clark_origin_2022}
\CSLLeftMargin{3. }%
\CSLRightInline{J. W. Clark, B. J. Harris, A. J. Hetherington, N.
Hurtado-Castano, R. A. Brench, S. Casson, T. A. Williams, J. E. Gray, A.
M. Hetherington, \href{https://doi.org/10.1016/j.cub.2022.04.040}{The
origin and evolution of stomata}. \emph{Current Biology} \textbf{32},
R539--R553 (2022).}

\bibitem[\citeproctext]{ref-hetherington_role_2003}
\CSLLeftMargin{4. }%
\CSLRightInline{A. M. Hetherington, F. I. Woodward,
\href{https://doi.org/10.1038/nature01843}{The role of stomata in
sensing and driving environmental change}. \emph{Nature} \textbf{424},
901--908 (2003).}

\bibitem[\citeproctext]{ref-de_boer_optimal_2016}
\CSLLeftMargin{5. }%
\CSLRightInline{H. J. de Boer, C. A. Price, F. Wagner‐Cremer, S. C.
Dekker, P. J. Franks, E. J. Veneklaas,
\href{https://doi.org/10.1111/nph.13929}{Optimal allocation of leaf
epidermal area for gas exchange}. \emph{New Phytologist} \textbf{210},
1219--1228 (2016).}

\bibitem[\citeproctext]{ref-harrison_influence_2020}
\CSLLeftMargin{6. }%
\CSLRightInline{E. L. Harrison, L. Arce Cubas, J. E. Gray, C. Hepworth,
\href{https://doi.org/10.1111/tpj.14560}{The influence of stomatal
morphology and distribution on photosynthetic gas exchange}. \emph{The
Plant Journal} \textbf{101}, 768--779 (2020).}

\bibitem[\citeproctext]{ref-woodward_stomatal_1987}
\CSLLeftMargin{7. }%
\CSLRightInline{F. I. Woodward,
\href{https://doi.org/10.1038/327617a0}{Stomatal numbers are sensitive
to increases in {CO2} from pre-industrial levels}. \emph{Nature}
\textbf{327}, 617--618 (1987).}

\bibitem[\citeproctext]{ref-buckley_modelling_2013}
\CSLLeftMargin{8. }%
\CSLRightInline{T. N. Buckley, K. A. Mott,
\href{https://doi.org/10.1111/pce.12140}{Modelling stomatal conductance
in response to environmental factors: {Modelling} stomatal conductance}.
\emph{Plant, Cell \& Environment} \textbf{36}, 1691--1699 (2013).}

\bibitem[\citeproctext]{ref-haworth_co-ordination_2013}
\CSLLeftMargin{9. }%
\CSLRightInline{M. Haworth, C. Elliott-Kingston, J. C. McElwain,
\href{https://doi.org/10.1007/s00442-012-2406-9}{Co-ordination of
physiological and morphological responses of stomata to elevated
{[}{CO2}{]} in vascular plants}. \emph{Oecologia} \textbf{171}, 71--82
(2013).}

\bibitem[\citeproctext]{ref-liang_stomatal_2023}
\CSLLeftMargin{10. }%
\CSLRightInline{X. Liang, D. Wang, Q. Ye, J. Zhang, M. Liu, H. Liu, K.
Yu, Y. Wang, E. Hou, B. Zhong, L. Xu, T. Lv, S. Peng, H. Lu, P. Sicard,
A. Anav, D. S. Ellsworth,
\href{https://doi.org/10.1038/s41467-023-37934-7}{Stomatal responses of
terrestrial plants to global change}. \emph{Nature Communications}
\textbf{14}, 2188 (2023).}

\bibitem[\citeproctext]{ref-chua_stomatal_2024}
\CSLLeftMargin{11. }%
\CSLRightInline{L. C. Chua, O. S. Lau,
\href{https://doi.org/10.1242/dev.202681}{Stomatal development in the
changing climate}. \emph{Development} \textbf{151}, dev202681 (2024).}

\bibitem[\citeproctext]{ref-lang_century-long_2024}
\CSLLeftMargin{12. }%
\CSLRightInline{P. L. M. Lang, J. M. Erberich, L. Lopez, C. L. Weiß, G.
Amador, H. F. Fung, S. M. Latorre, J. R. Lasky, H. A. Burbano, M.
Expósito-Alonso, D. C. Bergmann,
\href{https://doi.org/10.1038/s41559-024-02481-x}{Century-long timelines
of herbarium genomes predict plant stomatal response to climate change}.
\emph{Nature Ecology \& Evolution} \textbf{8}, 1641--1653 (2024).}

\bibitem[\citeproctext]{ref-franks_new_2014}
\CSLLeftMargin{13. }%
\CSLRightInline{P. J. Franks, D. L. Royer, D. J. Beerling, P. K. Van de
Water, D. J. Cantrill, M. M. Barbour, J. A. Berry,
\href{https://doi.org/10.1002/2014GL060457}{New constraints on
atmospheric {CO}\(_{\textrm{2}}\) concentration for the {Phanerozoic}}.
\emph{Geophysical Research Letters} \textbf{41}, 4685--4694 (2014).}

\bibitem[\citeproctext]{ref-mcelwain_paleoecology_2017}
\CSLLeftMargin{14. }%
\CSLRightInline{J. C. McElwain, M. Steinthorsdottir, Paleoecology,
ploidy, paleoatmospheric composition, and developmental biology: A
review of the multiple uses of fossil stomata. \emph{Plant Physiology}
\textbf{174}, 650--664 (2017).}

\bibitem[\citeproctext]{ref-the_cenozoic_co_proxy_integration_project_cencopip_consortium_toward_2023}
\CSLLeftMargin{15. }%
\CSLRightInline{The Cenozoic CO Proxy Integration Project (CenCOPIP)
Consortium*†, B. Hönisch, D. L. Royer, D. O. Breecker, P. J. Polissar,
G. J. Bowen, M. J. Henehan, Y. Cui, M. Steinthorsdottir, J. C. McElwain,
M. J. Kohn, A. Pearson, S. R. Phelps, K. T. Uno, A. Ridgwell, E.
Anagnostou, J. Austermann, M. P. S. Badger, R. S. Barclay, P. K. Bijl,
T. B. Chalk, C. R. Scotese, E. De La Vega, R. M. DeConto, K. A. Dyez, V.
Ferrini, P. J. Franks, C. F. Giulivi, M. Gutjahr, D. T. Harper, L. L.
Haynes, M. Huber, K. E. Snell, B. A. Keisling, W. Konrad, T. K.
Lowenstein, A. Malinverno, M. Guillermic, L. M. Mejía, J. N. Milligan,
J. J. Morton, L. Nordt, R. Whiteford, A. Roth-Nebelsick, J. K. C.
Rugenstein, M. F. Schaller, N. D. Sheldon, S. Sosdian, E. B. Wilkes, C.
R. Witkowski, Y. G. Zhang, L. Anderson, D. J. Beerling, C. Bolton, T. E.
Cerling, J. M. Cotton, J. Da, D. D. Ekart, G. L. Foster, D. R.
Greenwood, E. G. Hyland, E. A. Jagniecki, J. P. Jasper, J. B. Kowalczyk,
L. Kunzmann, W. M. Kürschner, C. E. Lawrence, C. H. Lear, M. A.
Martínez-Botí, D. P. Maxbauer, P. Montagna, B. D. A. Naafs, J. W. B.
Rae, M. Raitzsch, G. J. Retallack, S. J. Ring, O. Seki, J. Sepúlveda, A.
Sinha, T. F. Tesfamichael, A. Tripati, J. Van Der Burgh, J. Yu, J. C.
Zachos, L. Zhang, \href{https://doi.org/10.1126/science.adi5177}{Toward
a {Cenozoic} history of atmospheric {CO}\(_{\textrm{2}}\)}.
\emph{Science} \textbf{382}, eadi5177 (2023).}

\bibitem[\citeproctext]{ref-hofmann_impact_2025}
\CSLLeftMargin{16. }%
\CSLRightInline{T. A. Hofmann, W. Atkinson, M. Fan, A. J. Simkin, P.
Jindal, T. Lawson, \href{https://doi.org/10.1098/rstb.2024.0244}{Impact
of climate-driven changes in temperature on stomatal anatomy and
physiology}. \emph{Philosophical Transactions of the Royal Society B:
Biological Sciences} \textbf{380}, 20240244 (2025).}

\bibitem[\citeproctext]{ref-berry_stomata:_2010}
\CSLLeftMargin{17. }%
\CSLRightInline{J. A. Berry, D. J. Beerling, P. J. Franks,
\href{https://doi.org/10.1016/j.pbi.2010.04.013}{Stomata: Key players in
the earth system, past and present}. \emph{Current Opinion in Plant
Biology} \textbf{13}, 232--239 (2010).}

\bibitem[\citeproctext]{ref-franks_stomatal_2017}
\CSLLeftMargin{18. }%
\CSLRightInline{P. J. Franks, J. A. Berry, D. L. Lombardozzi, G. B.
Bonan, \href{https://doi.org/10.1104/pp.17.00287}{Stomatal {Function}
across {Temporal} and {Spatial} {Scales}: {Deep}-{Time} {Trends},
{Land}-{Atmosphere} {Coupling} and {Global} {Models}}. \emph{Plant
Physiology} \textbf{174}, 583--602 (2017).}

\bibitem[\citeproctext]{ref-grubb_leaf_1977}
\CSLLeftMargin{19. }%
\CSLRightInline{P. J. Grubb, {``Leaf structure and function''} in
\emph{The Encyclopedia of Ignorance}, R. Duncan, M. Weston-Smith, Eds.
(Pergamon, Oxford, 1977)vol. 2, pp. 317--330.}

\bibitem[\citeproctext]{ref-parkhurst_adaptive_1978}
\CSLLeftMargin{20. }%
\CSLRightInline{D. F. Parkhurst,
\href{https://doi.org/10.2307/2259142}{The adaptive significance of
stomatal occurrence on one or both surfaces of leaves}. \emph{The
Journal of Ecology} \textbf{66}, 367--383 (1978).}

\bibitem[\citeproctext]{ref-mott_adaptive_1982}
\CSLLeftMargin{21. }%
\CSLRightInline{K. A. Mott, A. C. Gibson, J. W. O'Leary,
\href{https://doi.org/10.1111/1365-3040.ep11611750}{The adaptive
significance of amphistomatic leaves}. \emph{Plant, Cell \& Environment}
\textbf{5}, 455--460 (1982).}

\bibitem[\citeproctext]{ref-gibson_structure-function_1996}
\CSLLeftMargin{22. }%
\CSLRightInline{A. C. Gibson, \emph{Structure-{Function} {Relations} of
{Warm} {Desert} {Plants}} (Springer Berlin / Heidelberg, Berlin,
Heidelberg, 1996;
\url{http://public.eblib.com/choice/PublicFullRecord.aspx?p=6495247}).}

\bibitem[\citeproctext]{ref-smith_leaf_1997}
\CSLLeftMargin{23. }%
\CSLRightInline{W. K. Smith, T. C. Vogelmann, E. H. DeLucia, D. T. Bell,
K. A. Shepherd, \href{https://doi.org/10.2307/1313100}{Leaf {Form} and
{Photosynthesis}}. \emph{BioScience} \textbf{47}, 785--793 (1997).}

\bibitem[\citeproctext]{ref-oguchi_leaf_2018}
\CSLLeftMargin{24. }%
\CSLRightInline{R. Oguchi, Y. Onoda, I. Terashima, D. Tholen, {``Leaf
{Anatomy} and {Function}''} in \emph{The {Leaf}: {A} {Platform} for
{Performing} {Photosynthesis}}, W. W. Adams III, I. Terashima, Eds.
(Springer International Publishing, Cham, 2018;
\url{https://doi.org/10.1007/978-3-319-93594-2_5})\emph{Advances in
{Photosynthesis} and {Respiration}}, pp. 97--139.}

\bibitem[\citeproctext]{ref-drake_two_2019}
\CSLLeftMargin{25. }%
\CSLRightInline{P. L. Drake, H. J. de Boer, S. J. Schymanski, E. J.
Veneklaas, \href{https://doi.org/10.1111/nph.15652}{Two sides to every
leaf: Water and {CO}\(_{\textrm{2}}\) transport in hypostomatous and
amphistomatous leaves}. \emph{New Phytologist} \textbf{222}, 1179--1187
(2019).}

\bibitem[\citeproctext]{ref-grubb_leaf_2020}
\CSLLeftMargin{26. }%
\CSLRightInline{P. J. Grubb, {``Leaf structure and function''} in
\emph{Unsolved {Problems} in {Ecology}}, A. Dobson, D. Tilman, R. D.
Holt, Eds. (Princeton University Press, Princeton, 2020), pp. 124--144.}

\bibitem[\citeproctext]{ref-pospisilova_environmental_1980}
\CSLLeftMargin{27. }%
\CSLRightInline{J. Pospíŝilová, J. Solárová, Environmental and
biological control of diffusive conductances of adaxial and abaxial leaf
epidermes. \emph{Photosynthetica} \textbf{14}, 90--127 (1980).}

\bibitem[\citeproctext]{ref-mott_stomatal_1984}
\CSLLeftMargin{28. }%
\CSLRightInline{K. A. Mott, J. W. O'Leary,
\href{http://www.jstor.org/stable/4268406}{Stomatal {Behavior} and {CO}₂
{Exchange} {Characteristics} in {Amphistomatous} {Leaves}}. \emph{Plant
Physiology} \textbf{74}, 47--51 (1984).}

\bibitem[\citeproctext]{ref-reich_effects_1985}
\CSLLeftMargin{29. }%
\CSLRightInline{P. B. Reich, A. W. Schoettle, R. G. Amundson,
\href{https://doi.org/10.1111/j.1399-3054.1985.tb02818.x}{Effects of low
concentrations of {O3}, leaf age and water stress on leaf diffusive
conductance and water use efficiency in soybean}. \emph{Physiologia
Plantarum} \textbf{63}, 58--64 (1985).}

\bibitem[\citeproctext]{ref-mott_asymmetric_1993}
\CSLLeftMargin{30. }%
\CSLRightInline{K. A. Mott, Z. G. Cardon, J. A. Berry,
\href{https://doi.org/10.1111/j.1365-3040.1993.tb00841.x}{Asymmetric
patchy stomatal closure for the two surfaces of \emph{{Xanthium}
strumarium} {L}. Leaves at low humidity}. \emph{Plant, Cell \&
Environment} \textbf{16}, 25--34 (1993).}

\bibitem[\citeproctext]{ref-wall_stomata_2022}
\CSLLeftMargin{31. }%
\CSLRightInline{S. Wall, S. Vialet‐Chabrand, P. Davey, J. Van Rie, A.
Galle, J. Cockram, T. Lawson,
\href{https://doi.org/10.1111/nph.18257}{Stomata on the abaxial and
adaxial leaf surfaces contribute differently to leaf gas exchange and
photosynthesis in wheat}. \emph{New Phytologist} \textbf{235},
1743--1756 (2022).}

\bibitem[\citeproctext]{ref-gutschick_photosynthesis_1984}
\CSLLeftMargin{32. }%
\CSLRightInline{V. P. Gutschick, Photosynthesis model for
{C}\(_{\textrm{3}}\) leaves incorporating {CO}\(_{\textrm{2}}\)
transport, propagation of radiation, and biochemistry 2. Ecological and
agricultural utility. \emph{Photosynthetica} \textbf{18}, 569--595
(1984).}

\bibitem[\citeproctext]{ref-marquez_assessing_2023}
\CSLLeftMargin{33. }%
\CSLRightInline{D. A. Márquez, H. Stuart‐Williams, L. A. Cernusak, G. D.
Farquhar, \href{https://doi.org/10.1111/nph.18784}{Assessing the
{\textless{}}span style="font-variant:small-caps;"{\textgreater{}}
{CO}\(_{\textrm{2}}\) {\textless{}}/span{\textgreater{}} concentration
at the surface of photosynthetic mesophyll cells}. \emph{New
Phytologist} \textbf{238}, 1446--1460 (2023).}

\bibitem[\citeproctext]{ref-parkhurst_intercellular_1990}
\CSLLeftMargin{34. }%
\CSLRightInline{D. F. Parkhurst, K. A. Mott,
\href{https://doi.org/10.1104/pp.94.3.1024}{Intercellular diffusion
limits to {CO}\(_{\textrm{2}}\) uptake in leaves: Studies in air and
helox}. \emph{Plant Physiology} \textbf{94}, 1024--1032 (1990).}

\bibitem[\citeproctext]{ref-foster_influence_1986}
\CSLLeftMargin{35. }%
\CSLRightInline{J. R. Foster, W. K. Smith,
\href{https://doi.org/10.1111/j.1365-3040.1986.tb02108.x}{Influence of
stomatal distribution on transpiration in low-wind environments}.
\emph{Plant, Cell and Environment} \textbf{9}, 751--759 (1986).}

\bibitem[\citeproctext]{ref-muir_is_2019}
\CSLLeftMargin{36. }%
\CSLRightInline{C. D. Muir, \href{https://doi.org/10.1093/icb/icz085}{Is
amphistomy an adaptation to high light? {Optimality} models of stomatal
traits along light gradients}. \emph{Integrative and Comparative
Biology} \textbf{59}, 571--584 (2019).}

\bibitem[\citeproctext]{ref-wood_physiology_1934}
\CSLLeftMargin{37. }%
\CSLRightInline{J. G. Wood, The physiology of xerophytism in
{Australian} plants: The stomatal frequencies, transpiration and osmotic
pressures of sclerophyll and tomentose-succulent leaved plants.
\emph{Journal of Ecology} \textbf{22}, 69--87 (1934).}

\bibitem[\citeproctext]{ref-howell_concerning_1945}
\CSLLeftMargin{38. }%
\CSLRightInline{J. T. Howell, Concerning stomata on leaves in
\emph{a}rctostaphylos. \emph{The Wasmann Collector} \textbf{6}, 57--65
(1945).}

\bibitem[\citeproctext]{ref-salisbury_i_1928}
\CSLLeftMargin{39. }%
\CSLRightInline{E. J. Salisbury,
\href{https://doi.org/10.1098/rstb.1928.0001}{I. {On} the causes and
ecological significance of stomatal frequency, with special reference to
the woodland flora}. \emph{Philosophical Transactions of the Royal
Society of London. Series B, Containing Papers of a Biological
Character} \textbf{216}, 1--65 (1928).}

\bibitem[\citeproctext]{ref-peat_comparative_1994}
\CSLLeftMargin{40. }%
\CSLRightInline{H. J. Peat, A. H. Fitter, A comparative study of the
distribution and density of stomata in the {British} flora.
\emph{Biological Journal of the Linnean Society} \textbf{52}, 377--393
(1994).}

\bibitem[\citeproctext]{ref-jordan_using_2014}
\CSLLeftMargin{41. }%
\CSLRightInline{G. J. Jordan, R. J. Carpenter, T. J. Brodribb,
\href{https://doi.org/10.1016/j.palaeo.2013.12.035}{Using fossil leaves
as evidence for open vegetation}. \emph{Palaeogeography,
Palaeoclimatology, Palaeoecology} \textbf{395}, 168--175 (2014).}

\bibitem[\citeproctext]{ref-bucher_stomatal_2017}
\CSLLeftMargin{42. }%
\CSLRightInline{S. F. Bucher, K. Auerswald, C. Grün-Wenzel, S. I.
Higgins, J. Garcia Jorge, C. Römermann,
\href{https://doi.org/10.1016/j.flora.2017.02.011}{Stomatal traits
relate to habitat preferences of herbaceous species in a temperate
climate}. \emph{Flora} \textbf{229}, 107--115 (2017).}

\bibitem[\citeproctext]{ref-muir_light_2018}
\CSLLeftMargin{43. }%
\CSLRightInline{C. D. Muir, Light and growth form interact to shape
stomatal ratio among {British} angiosperms. \emph{New Phytologist}
\textbf{218}, 242--252 (2018).}

\bibitem[\citeproctext]{ref-triplett_stomatal_2025}
\CSLLeftMargin{44. }%
\CSLRightInline{G. Triplett, A. S. David,
\href{https://doi.org/10.1002/ajb2.70050}{Stomatal distribution and
post‐fire recovery: {Intra}‐ and interspecific variation in plants of
the pyrogenic {Florida} scrub}. \emph{American Journal of Botany},
e70050 (2025).}

\bibitem[\citeproctext]{ref-lyshede_comparative_2002}
\CSLLeftMargin{45. }%
\CSLRightInline{O. B. Lyshede, Comparative and functional leaf anatomy
of selected {Alstroemeriaceae} of mainly {Chilean} origin.
\emph{Botanical Journal of the Linnean Society} \textbf{140}, 261--272
(2002).}

\bibitem[\citeproctext]{ref-triplett_amphistomy_2024}
\CSLLeftMargin{46. }%
\CSLRightInline{G. Triplett, T. N. Buckley, C. D. Muir,
\href{https://doi.org/10.1002/ajb2.16284}{Amphistomy increases leaf
photosynthesis more in coastal than montane plants of {Hawaiian} ʻilima
(\emph{{Sida} fallax})}. \emph{American Journal of Botany} \textbf{111},
e16284 (2024).}

\bibitem[\citeproctext]{ref-farquhar_biochemical_1980}
\CSLLeftMargin{47. }%
\CSLRightInline{G. D. Farquhar, S. von Caemmerer, J. A. Berry,
\href{https://doi.org/10.1007/BF00386231}{A biochemical model of
photosynthetic {CO}\(_{\textrm{2}}\) assimilation in leaves of
{C}\(_{\textrm{3}}\) species}. \emph{Planta} \textbf{149}, 78--90
(1980).}

\bibitem[\citeproctext]{ref-poorter_metaanalysis_2019}
\CSLLeftMargin{48. }%
\CSLRightInline{H. Poorter, Ü. Niinemets, N. Ntagkas, A. Siebenkäs, M.
Mäenpää, S. Matsubara, T. L. Pons,
\href{https://doi.org/10.1111/nph.15754}{A meta‐analysis of plant
responses to light intensity for 70 traits ranging from molecules to
whole plant performance}. \emph{New Phytologist} \textbf{223},
1073--1105 (2019).}

\bibitem[\citeproctext]{ref-givnish_common-garden_2014}
\CSLLeftMargin{49. }%
\CSLRightInline{T. J. Givnish, R. A. Montgomery,
\href{https://doi.org/10.1098/rspb.2013.2944}{Common-garden studies on
adaptive radiation of photosynthetic physiology among {Hawaiian}
lobeliads}. \emph{Proceedings of the Royal Society B: Biological
Sciences} \textbf{281}, 20132944--20132944 (2014).}

\bibitem[\citeproctext]{ref-engelbrecht_evaluation_2001}
\CSLLeftMargin{50. }%
\CSLRightInline{B. M. J. Engelbrecht, H. M. Herz,
\href{https://doi.org/10.1017/S0266467401001146}{Evaluation of different
methods to estimate understorey light conditions in tropical forests}.
\emph{Journal of Tropical Ecology} \textbf{17}, 207--224 (2001).}

\bibitem[\citeproctext]{ref-peralta_taxonomy_2008}
\CSLLeftMargin{51. }%
\CSLRightInline{I. E. Peralta, D. M. Spooner, S. Knapp, Taxonomy of wild
tomatoes and their relatives (\emph{solanum} sect.
\emph{Lycopersicoides}, sect. \emph{Juglandifolia}, sect.
\emph{Lycopersicon}; {Solanaceae}). \textbf{84} (2008).}

\bibitem[\citeproctext]{ref-burns_multi-resolution_2024}
\CSLLeftMargin{52. }%
\CSLRightInline{P. Burns, C. R. Hakkenberg, S. J. Goetz,
\href{https://doi.org/10.1038/s41597-024-03668-4}{Multi-resolution
gridded maps of vegetation structure from {GEDI}}. \emph{Scientific
Data} \textbf{11}, 881 (2024).}

\bibitem[\citeproctext]{ref-schoch_dependence_1980}
\CSLLeftMargin{53. }%
\CSLRightInline{P.-G. Schoch, C. Zinsou, M. Sibi,
\href{https://doi.org/10.1093/jxb/31.5.1211}{Dependence of the stomatal
index on environmental factors during stomatal differentiation in leaves
of \emph{{Vigna} sinensis} {L}.: 1. {Effect} of light intensity}.
\emph{Journal of Experimental Botany} \textbf{31}, 1211--1216 (1980).}

\bibitem[\citeproctext]{ref-sack_developmental_2016}
\CSLLeftMargin{54. }%
\CSLRightInline{L. Sack, T. N. Buckley,
\href{https://doi.org/10.1104/pp.16.00476}{The developmental basis of
stomatal density and flux}. \emph{Plant Physiology} \textbf{171},
2358--2363 (2016).}

\bibitem[\citeproctext]{ref-mott_amphistomy_1991}
\CSLLeftMargin{55. }%
\CSLRightInline{K. A. Mott, O. Michaelson, Amphistomy as an adaptation
to high light intensity in \emph{{Ambrosia} cordifolia} ({Compositae}).
\emph{American Journal of Botany} \textbf{78}, 76--79 (1991).}

\bibitem[\citeproctext]{ref-schindelin_fiji_2012}
\CSLLeftMargin{56. }%
\CSLRightInline{J. Schindelin, I. Arganda-Carreras, E. Frise, V. Kaynig,
M. Longair, T. Pietzsch, S. Preibisch, C. Rueden, S. Saalfeld, B.
Schmid, J.-Y. Tinevez, D. J. White, V. Hartenstein, K. Eliceiri, P.
Tomancak, A. Cardona, \href{https://doi.org/10.1038/nmeth.2019}{Fiji: An
open-source platform for biological-image analysis}. \emph{Nature
Methods} \textbf{9}, 676--682 (2012).}

\bibitem[\citeproctext]{ref-marshall_model_1980}
\CSLLeftMargin{57. }%
\CSLRightInline{B. Marshall, P. V. Biscoe,
\href{https://doi.org/10.1093/jxb/31.1.29}{A {Model} for
{C}\(_{\textrm{3}}\) {Leaves} {Describing} the {Dependence} of {Net}
{Photosynthesis} on {Irradiance}}. \emph{Journal of Experimental Botany}
\textbf{31}, 29--39 (1980).}

\bibitem[\citeproctext]{ref-elzhov_minpacklm_2023}
\CSLLeftMargin{58. }%
\CSLRightInline{T. V. Elzhov, K. M. Mullen, A.-N. Spiess, B. M. Bolker,
\emph{Minpack.lm: {R} {Interface} to the {Levenberg}-{Marquardt}
{Nonlinear} {Least}-{Squares} {Algorithm} {Found} in {MINPACK}, {Plus}
{Support} for {Bounds}} (2023;
\url{https://CRAN.R-project.org/package=minpack.lm}).}

\bibitem[\citeproctext]{ref-stan_development_team_stan_2025}
\CSLLeftMargin{59. }%
\CSLRightInline{Stan Development Team, \emph{Stan {Modeling} {Language}
{Users} {Guide} and {Reference} {Manual}} (2025;
\url{https://mc-stan.org}).}

\bibitem[\citeproctext]{ref-burkner_brms_2017}
\CSLLeftMargin{60. }%
\CSLRightInline{P.-C. Bürkner,
\href{https://doi.org/10.18637/jss.v080.i01}{\textbf{Brms} : {An}
\emph{r} {Package} for {Bayesian} {Multilevel} {Models} {Using}
\emph{stan}}. \emph{Journal of Statistical Software} \textbf{80}
(2017).}

\bibitem[\citeproctext]{ref-gabry_cmdstanr_2025}
\CSLLeftMargin{61. }%
\CSLRightInline{J. Gabry, R. Češnovar, A. Johnson, S. Bronder,
\emph{Cmdstanr: {R} {Interface} to '{CmdStan}'} (2025;
\href{https://mc-stan.org/cmdstanr,\%20https://discourse.mc-stan.org}{https://mc-stan.org/cmdstanr,
https://discourse.mc-stan.org}).}

\bibitem[\citeproctext]{ref-r_core_team_r:_2025}
\CSLLeftMargin{62. }%
\CSLRightInline{R Core Team, \emph{R: {A} {Language} and {Environment}
for {Statistical} {Computing}} (R Foundation for Statistical Computing,
Vienna, Austria, 2025; \url{http://www.R-project.org/}).}

\bibitem[\citeproctext]{ref-gelman_inference_1992}
\CSLLeftMargin{63. }%
\CSLRightInline{A. Gelman, D. B. Rubin, Inference from iterative
simulation using multiple sequences. \emph{Statistical Science}
\textbf{7}, 457--472 (1992).}

\bibitem[\citeproctext]{ref-hansen_stabilizing_1997}
\CSLLeftMargin{64. }%
\CSLRightInline{T. F. Hansen,
\href{https://doi.org/10.1111/j.1558-5646.1997.tb01457.x}{Stabilizing
selection and the comparative analysis of adaptation}. \emph{Evolution}
\textbf{51}, 1341--1351 (1997).}

\bibitem[\citeproctext]{ref-vehtari_practical_2017}
\CSLLeftMargin{65. }%
\CSLRightInline{A. Vehtari, A. Gelman, J. Gabry,
\href{https://doi.org/10.1007/s11222-016-9696-4}{Practical {Bayesian}
model evaluation using leave-one-out cross-validation and {WAIC}}.
\emph{Statistics and Computing} \textbf{27}, 1413--1432 (2017).}

\end{CSLReferences}



\end{document}
